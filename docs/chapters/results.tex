\chapter{Podsumowanie}
\label{chap:results}

	Niniejszy rozdział dokonuje podsumowania rezultatów prowadzonej pracy. Porównuje szacunki dotyczące poboru mocy z rzeczywistym poborem. Ponadto dokonuje oceny bezpieczeństwa oraz responsywności systemu.

	W ramach niniejszej pracy stworzono funkcjonalny prototyp systemu, który po odpowiednich modyfikacjach miałby szansę efektywnej pracy w odpowiednim środowisku.

	Dzięki wykorzystaniu zdalnego serwera do przeprowadzenia procesu uwierzytelniania system zapewnia większą elastyczność i łatwość zarządzania niż alternatywne systemy wykorzystujące zamki pracujące w sposób autonomiczny.

	Rozwiązanie cechuje się wygodą montażu, ponieważ nie wymaga przewodów zasilających i komunikacyjnych prowadzonych w ścianach budynków. Przy wdrażaniu rozwiązania nie jest konieczna modyfikacja istniejącej infrastruktury budynku, z wyjątkiem wymiany samych zamków. System nie wymaga żadnych dodatkowych komponentów sprzętowych poza zamkami i serwerem.

	Wydajność energetyczna podsystemu sterowania zamkiem została osiągnięta przez zarządzanie zasilaniem jego peryferiów oraz kontrolę stanu zasilania mikrokontrolera w celu minimalizacji poboru mocy i maksymalizacji czasu pracy na zasilaniu bateryjnym.

	Bezpieczeństwo systemu na wielu poziomach zapewnia wykorzystanie mechanizmów takich jak TLS w warstwie komunikacji pomiędzy zamkiem a serwerem czy szyfrowanie pamięci Flash w warstwie operacji na danych w mikroprocesorze w układzie zamka.

	\textbf{Wydajność energetyczna - TBD}

	\textbf{Responsywność - TBD}

	\textbf{Bezpieczeństwo - TBD}

	\section{Możliwe rozszerzenia}

        W ramach niniejszej pracy stworzony został prototyp rozwiązania. Niektóre planowane funkcjonalności nie zostały zaimplementowane ze względu na ograniczenia czasowe i budżetowe. Pozostawiono jednak możliwość rozbudowy systemu. Poniżej przedstawiono kilka problemów, których system w obecnym stanie nie adresuje, wraz z możliwymi rozwiązaniami.

        \subsection{Mechanizm wyjścia}

            Opisywane rozwiązanie nie obejmuje implementacji mechanizmu opuszczenia strefy chronionej systemem kontroli dostępu. W zależności od potrzeb końcowego użytkownika, możliwe rozwiązanie to montaż dodatkowego czytnika po przeciwnej stronie drzwi i połączenie go z kontrolerem wejścia w przypadku gdy wymagana jest obustronna kontrola dostępu bądź zastosowanie przycisku którego naciśnięcie powoduje zwolnienie zamka w przypadku gdy wymagana jest tylko kontrola wejścia do chronionego obszaru.

        \subsection{Obsługa większej liczby zamków}

            W celu umożliwienia obsługi przez system liczby zamków przekraczającej 1, wystarczająca byłaby modyfikacja podsystemu autoryzacji w taki sposób, aby mógł on obsługiwać równoległe żądania od klientów.

        \subsection{Wygodna konfiguracja parametrów sieci}

            W obecnej implementacji dane dostępu do sieci (nazwa sieci oraz hasło) zostały zagnieżdżone w oprogramowaniu konrolera. Zmniejsza to elastyczność konfiguracji urządzenia, wymagając jego przeprogramowania za każdym razem gdy zmianie ulegnie nazwa lub klucz dostępu do sieci.

            Możliwym rozwiązaniem tego problemu byłaby implementacja trybu konfiguracji. Tryb ten powodowałby przejście kontrolera w tryb Access Point przy zachowaniu dwóch warunków: (1) nastąpiło uruchomienie, a nie wybudzenie z trybu głębokiego uśpienia oraz (2) na określonym wejściu pojawił się stan wysoki. Przejście kontrolera w tryb Access Point umożliwiłoby udostępnienie prostego interfejsu webowego, za pomocą którego administrator systemu mógłby wprowadzić niezbędne do działania dane, takie jak nazwa sieci, hasło, a także adres IP i numer portu serwera autoryzacji. Aby zachować wysoki poziom bezpieczeństwa, komunikacja pomiędzy urządzeniem administratora i kontrolerem powinna odbywać się przy wykorzystaniu protokołu TLS.

        \subsection{Sygnalizacja stanu baterii}

            W obecnym stanie system nie implementuje mechanizmów informowania serwera o swoim stanie energetycznym.
            Sygnalizacja stanu baterii umożliwiłaby administratorowi systemu bieżące monitorowanie wszystkich zamków objętych systemem oraz szybką reakcję w przypadku, gdy baterie wymagałyby wymiany. Modyfikacja ta wymagałaby uzyskania dostępu do danych na temat naładowania baterii przez mikrokontroler oraz przesyłania ich okresowo do serwera, najlepiej w momentach, gdy jest on już wybudzony z powodu wykrycia ruchu w jego otoczeniu.

        \subsection{Obsługa kont użytkowników w podsystemie zarządzania}
            Podsystem zarządzania nie implementuje mechanizmu dostępu do zasobów, co czyni system mniej bezpiecznym. W końcowym produkcie należałoby rozszerzyć go o możliwość tworzenia kont użytkowników i przypisywania im określonych uprawnień co do odczytu i zapisu danych.
