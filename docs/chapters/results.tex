\chapter{Podsumowanie}
\label{chap:results}

	Niniejszy rozdział dokonuje podsumowania rezultatów prowadzonej pracy. Porównuje szacunki dotyczące poboru mocy z rzeczywistym poborem. Ponadto dokonuje oceny bezpieczeństwa oraz responsywności systemu.

	W ramach niniejszej pracy stworzono funkcjonalny prototyp systemu, który po odpowiednich modyfikacjach miałby szansę efektywnej pracy w odpowiednim środowisku.

	Dzięki wykorzystaniu zdalnego serwera do przeprowadzenia procesu uwierzytelniania system zapewnia większą elastyczność i łatwość zarządzania niż alternatywne systemy wykorzystujące zamki pracujące w sposób autonomiczny.

	Rozwiązanie cechuje się wygodą montażu, ponieważ nie wymaga przewodów zasilających i komunikacyjnych prowadzonych w ścianach budynków. Przy wdrażaniu rozwiązania nie jest konieczna modyfikacja istniejącej infrastruktury budynku, z wyjątkiem wymiany samych zamków. System nie wymaga żadnych dodatkowych komponentów sprzętowych poza zamkami i serwerem.

	Wydajność energetyczna podsystemu sterowania zamkiem została osiągnięta przez zarządzanie zasilaniem jego peryferiów oraz kontrolę stanu zasilania mikrokontrolera w celu minimalizacji poboru mocy i maksymalizacji czasu pracy na zasilaniu bateryjnym.

	Bezpieczeństwo systemu na wielu poziomach zapewnia wykorzystanie mechanizmów takich jak TLS w warstwie komunikacji pomiędzy zamkiem a serwerem czy szyfrowanie pamięci Flash w warstwie operacji na danych w mikroprocesorze w układzie zamka.

	\textbf{Wydajność energetyczna - TBD}

	\textbf{Responsywność - TBD}

	\textbf{Bezpieczeństwo - TBD}