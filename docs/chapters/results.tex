\newchapter{Podsumowanie}{Adrianna Piekarska}
\label{chap:results}

	Niniejszy rozdział dokonuje podsumowania rezultatów prowadzonej pracy. W pierwszej części przedstawia podział prac i obowiązków, który przyjęto w ramach projektu. Następnie prezentuje opisy i wyniki testów, które przeprowadzono w celu weryfikacji spełnienia wymagań, po czym przechodzi do porównania wstępnych szacunków dotyczących poboru mocy z rzeczywistym poborem. W kolejnej części przedstawia możliwe rozszerzenia powstałego prototypu. Ostatnia część rozdziału jest podsumowaniem pracy.

    \section{Podział prac}

    Tabela \ref{tbl:podzial} przedstawia podział odpowiedzialności za poszczególne elementy projektu.

     \begin{table}[]
            \caption{Podział odpowiedzialności}
            \centering
            \begin{tabular}{p{8cm}|p{4cm}}
            \textbf{Element} & \textbf{Osoba odpowiedzialna} \\ \hline
            Oprogramowanie mikrokontrolera w układzie zamka & Adrianna Piekarska, Grzegorz Wąs \\
            \hline
            Układ zasilający czytnik RFID & Grzegorz Wąs \\
            \hline
            Oprogramowanie autoryzujące & Grzegorz Wąs \\
            \hline
            Oprogramowanie zarządzające - Aplikacja API & Adrianna Piekarska \\
            \hline
            Oprogramowanie zarządzające - Aplikacja GUI & Adrianna Piekarska \\
            \hline
            Projekt i utworzenie bazy danych & Adrianna Piekarska \\
            \hline
            Przygotowanie środowiska uruchomieniowego projektu & Grzegorz Wąs \\
            \end{tabular}
            \label{tbl:podzial}
    \end{table}



    \section{Testowanie}

        Tabela \ref{tbl:tests} przedstawia przeprowadzone testy oraz ich wyniki. Każdy test zawiera odniesienie do wymagania, które pokrywa.

        \begin{table}[h!]
            \caption{Testy wymagań}
            \centering
            \begin{subtable}[c]{\textwidth}
                \centering
                    \begin{tabular}{p{2cm}|p{2,5cm}|p{9,5cm}}
                    TEST\_1               & \multicolumn{2}{l}{\textbf{Poprawność procesu autoryzacji}} \\ \hline
                    Wymaganie             & \multicolumn{2}{p{12cm}}{FNRQ\_1 Kontrola dostępu do pomieszczeń, FNRQ\_8 Sygnalizacja przyznania lub odmowy dostępu} \\ \hline
                    \multirow{2}{*}{Opis} & Przebieg           & Zbliżenie się do zamka, przyłożenie identyfikatora upoważnionego do uzyskania dostępu do zamka. \\ \cline{2-3}
                                          & Oczekiwany wynik   & Układ zamka informuje o pomyślnej decyzji.                                                 \\ \hline
                    Wynik                 & \multicolumn{2}{p{12cm}}{Pozytywny - układ zamka informuje o pomyślnej decyzji.}                                                                                  \\
                    \end{tabular}%
                \label{tbl:test1}
                \vspace{10mm}
            \end{subtable}
        \quad%
            \begin{subtable}[c]{\textwidth}
                \centering
                    \begin{tabular}{p{2cm}|p{2,5cm}|p{9,5cm}}
                    TEST\_2               & \multicolumn{2}{l}{\textbf{Poprawność procesu autoryzacji}}                                                            \\ \hline
                    Wymaganie             & \multicolumn{2}{p{12cm}}{FNRQ\_1 Kontrola dostępu do pomieszczeń, FNRQ\_8 Sygnalizacja przyznania lub odmowy dostępu} \\ \hline
                    \multirow{2}{*}{Opis} & Przebieg           & Zbliżenie się do zamka, przyłożenie identyfikatora nieupoważnionego do uzyskania dostępu do zamka. \\ \cline{2-3}
                                          & Oczekiwany wynik   & Układ zamka informuje o niepomyślnej decyzji.                                                 \\ \hline
                    Wynik                 & \multicolumn{2}{p{12cm}}{Pozytywny - układ zamka informuje o niepomyślnej decyzji. Przetestowano z użyciem dwóch kart RFID oraz jednego breloka RFID.}
                    \end{tabular}%
                \label{tbl:test2}
                \vspace{10mm}
            \end{subtable}
        \quad%
            \begin{subtable}[c]{\textwidth}
                \centering
                    \begin{tabular}{p{2cm}|p{2,5cm}|p{9,5cm}}
                    TEST\_3               & \multicolumn{2}{l}{\textbf{Poprawność zapisu i podglądu historii dokonanych prób dostępu}}                                                            \\ \hline
                    Wymaganie             & \multicolumn{2}{p{12cm}}{FNRQ\_2 Przeglądanie historii prób dostępu do pomieszczeń}                                                                                    \\ \hline
                    \multirow{2}{*}{Opis} & Przebieg           & Dokonanie próby dostępu zakończonej sukcesem. Dokonanie próby dostępu zakończonej porażką.                                    \\ \cline{2-3}
                                         & Oczekiwany wynik   & Próby dostępu zakończone sukcesem i porażką zapisane w bazie danych.                                                 \\ \hline
                    Wynik                 & \multicolumn{2}{p{12cm}}{Pozytywny - próby dostępu zakończone sukcesem i porażką zapisane w bazie danych. Możliwy podgląd danych z aplikacji zarządzającej.} \\
                    \end{tabular}%
                \label{tbl:test3}
                \vspace{10mm}
            \end{subtable}
            \label{tbl:tests}
        \end{table}

        \pagebreak

        \begin{table}[h!]
            \ContinuedFloat
            \caption{Testy wymagań, c.d.}
            \begin{subtable}[c]{\textwidth}
                \centering
                    \begin{tabular}{p{2cm}|p{2,5cm}|p{9,5cm}}
                    TEST\_4               & \multicolumn{2}{l}{\textbf{Poprawność dodawania identyfikatorów}}                                                            \\ \hline
                    Wymaganie             & \multicolumn{2}{p{12cm}}{FNRQ\_3 Dodawanie identyfikatorów, FNRQ\_4 Przeglądanie identyfikatorów powiązanych z danym zamkiem, FNRQ\_6 Dostęp do grupy pomieszczeń za pomocą jednego identyfikatora }                                                                                    \\ \hline
                    \multirow{2}{*}{Opis} & Przebieg           & Dodanie identyfikatora za pomocą menu w aplikacji GUI, wraz z wyborem zamków, do których będzie upoważniony.  \\ \cline{2-3}
                                         & Oczekiwany wynik   & Identyfikator oraz powiązania z zamkami zapisane w bazie danych.                                                 \\ \hline
                    Wynik                 & \multicolumn{2}{p{12cm}}{Pozytywny - identyfikator oraz powiązania z zamkami zapisane w bazie danych. Możliwy podgląd danych z aplikacji zarządzającej.} \\
                    \end{tabular}%
                \label{tbl:test4}
                \vspace{10mm}
            \end{subtable}
        \quad%
            \begin{subtable}[c]{\textwidth}
                \centering
                    \begin{tabular}{p{2cm}|p{2,5cm}|p{9,5cm}}
                    TEST\_5               & \multicolumn{2}{p{12cm}}{\textbf{Poprawność blokowania dostępu do pomieszczenia dla żądanego identyfikatora}}                                                            \\ \hline
                    Wymaganie             & \multicolumn{2}{p{12cm}}{FNRQ\_5 Blokowanie dostępu do pomieszczeń dla wybranego identyfikatora }                                                                                    \\ \hline
                    \multirow{2}{*}{Opis} & Przebieg           & Usunięcie powiązania identyfikatora z zamkiem za pomocą przycisku w aplikacji GUI.  \\ \cline{2-3}
                                          & Oczekiwany wynik   & Żądane powiązanie usunięte z bazy danych. Dostęp za pomocą żądanego identyfikatora do żądanego zamka niemożliwy.                                                 \\ \hline
                    Wynik                 & \multicolumn{2}{p{12cm}}{Pozytywny - żądane powiązanie usunięte z bazy danych. Dostęp za pomocą żądanego identyfikatora do żądanego zamka niemożliwy.} \\
                    \end{tabular}%
                \label{tbl:test5}
                \vspace{10mm}
            \end{subtable}
        \quad%
            \begin{subtable}[c]{\textwidth}
                \centering
                    \begin{tabular}{p{2cm}|p{2,5cm}|p{9,5cm}}
                    TEST\_6               & \multicolumn{2}{p{12cm}}{\textbf{Dostęp do pomieszczenia za pomocą grupy identyfikatorów}}                                                            \\ \hline
                    Wymaganie             & \multicolumn{2}{p{12cm}}{FNRQ\_7 Dostęp do pomieszczenia za pomocą grupy identyfikatorów }                                                                                    \\ \hline
                    \multirow{2}{*}{Opis} & Przebieg           & Dodanie dwóch identyfikatorów za pomocą aplikacji GUI. Dodanie upoważnienia do tego samego zamka przy obu identyfikatorach.  \\ \cline{2-3}
                                         & Oczekiwany wynik   & Dwa powiązania zapisane w bazie danych. Obu identyfikatorom przyznawany jest dostęp.                                                 \\ \hline
                    Wynik                 & \multicolumn{2}{p{12cm}}{Pozytywny - dwa powiązania zapisane w bazie danych. Obu identyfikatorom przyznawany jest dostęp.} \\
                    \end{tabular}%
                \label{tbl:test6}
                \vspace{10mm}
            \end{subtable}
            \label{tbl:tests}
        \end{table}

        \pagebreak

        \begin{table}[h!]
            \ContinuedFloat
            \caption{Testy wymagań, c.d.}
            \begin{subtable}[c]{\textwidth}
                \centering
                    \begin{tabular}{p{2cm}|p{2,5cm}|p{9,5cm}}
                    TEST\_7               & \multicolumn{2}{l}{\textbf{Czas pracy układu zamka na zasilaniu bateryjnym}}                                                            \\ \hline
                    Wymaganie             & \multicolumn{2}{p{12cm}}{XXRQ\_1 Długość czasu pracy układu zamka na baterii równa minimum 1 rok }                                                                                    \\ \hline
                    \multirow{2}{*}{Opis} & Przebieg           & Przeprowadzenie pomiarów poboru prądu w uśpieniu i w trakcie działania układu. Wyliczenie żywotności na podstawie szacunków czasu działania układu i teoretycznej pojemności baterii\\ \cline{2-3}
                                          & Oczekiwany wynik   & Szacowana żywotność układu powinna wynosić minimum 1 rok                                                 \\ \hline
                    Wynik                 & \multicolumn{2}{p{12cm}}{Negatywny - Szacowana żywotność układu wynosi 30 dni}                                                                                  \\
                    \end{tabular}%
                \label{tbl:test6}
                \vspace{10mm}
            \end{subtable}
        \quad%
            \begin{subtable}[c]{\textwidth}
                \centering
                    \begin{tabular}{p{2cm}|p{2,5cm}|p{9,5cm}}
                    TEST\_8               & \multicolumn{2}{l}{\textbf{Czas odpowiedzi układu zamka}}                                                            \\ \hline
                    Wymaganie             & \multicolumn{2}{p{12cm}}{XXRQ\_2 Czas odpowiedzi układu zamka nie dłuższy niż 3 sekundy }                                                                                    \\ \hline
                    \multirow{2}{*}{Opis} & Przebieg           & Zbliżenie się do zamka, przyłożenie dowolnego identyfikatora do czytnika.  \\ \cline{2-3}
                                          & Oczekiwany wynik   & Układ zamka informuje o decyzji w czasie nie dłuższym niż 3 sekundy od momentu przyłożenia identyfikatora.                                                  \\ \hline
                    Wynik                 & \multicolumn{2}{p{12cm}}{\textbf{TODO: Wykonac test}} \\
                    \end{tabular}%
                \label{tbl:test6}
            \end{subtable}
            \label{tbl:tests}
        \end{table}

    \newsection{Wydajność energetyczna}{Grzegorz Wąs}
    \label{sec:wydajnosc}

        W celu przeprowadzenia szacunków dotyczących poboru prądu założono, że w wyniku prób dostępu lub przypadkowych wzbudzeń czujnika zbliżeniowego, mikrokontroler i czytnik RFID zostają wybudzone maksymalnie 120 razy w ciągu doby. Minimalny czas wybudzenia układu to 7 sekund, a maksymalny 16 sekund. Założono więc, że podczas każdego wybudzenia układ znajduje się w stanie operacyjnym przez okres czasu trwający od 7 do 16 sekund (średnio 11,5 sekundy). Wynika z tego, że przez ponad 98\% czasu pozostaje w stanie uśpienia. Dlatego pobór prądu w tym stanie będzie miał dominujący wpływ na ogólną efektywność energetyczną. Opisywane założenia przedstawiono w tabeli \ref{tbl:tab4}.

        \begin{table}[]
            \caption{Zakładane wartości dotyczące obciążenia układu}
            \centering
            \begin{tabular}{p{4cm}|p{3cm}|p{3cm}|p{3cm}}
                    \textbf{Średni czas przebywania układu w stanie aktywnym} & \textbf{Liczba wybudzeń w ciągu dnia} & \textbf{Łączny czas w ciągu dnia [s]} & \textbf{Łączny czas w ciągu dnia [\% doby]} \\ \hline
                     11,5 & 120 & 1380 & 1,60 \\
            \end{tabular}
            \label{tbl:tab4}
            \vspace{10mm}
        \end{table}

        Na podstawie wykonanych pomiarów można oszacować dobowy pobór prądu na około 107mAh. Stosując 9 ogniw galwanicznych o napięciu wyjściowym 1,5V i pojemności~1100mAh, połączonych zgodnie ze schematem przedstawionym na rysunku \ref{fig:battery_layout}, można zapewnić stałe źródło zasilania o napięciu 4,5V i pojemności 3300mAh. W takiej konfiguracji żywotność układu wynosi w przybliżeniu 30 dni.

        \begin{figure}[]
            \centering
            \includegraphics[width=0.5\textwidth]{chapters/images/battery_layout.png}
            \caption{Schemat połączenia baterii}
            \label{fig:battery_layout}
        \end{figure}

        Powodem stosunkowo niskiej żywotności układu jest wysoki pobór prądu w stanie spoczynku. Odpowiedzialny za tę nieefektywność jest element układu zasilania wykorzystanej płytki prototypowej, regulator napięcia LD1117. Jego standardowy prąd spoczynkowy wynosi około 5mA \cite{AMS1117-ds}. Zastępując go np. regulatorem AP2112K o prądzie spoczynkowym w granicach 50μA \cite{AP2112K-ds}, można zredukować pobór prądu w stanie uśpienia z 4mA do około 150μA, tym samym zwiększając żywotność. W tabeli \ref{tbl:tab1} przedstawiono porównanie rzeczywistych wartości poboru prądu z wartościami możliwymi do osiągnięcia w przypadku zastosowania regulatora napięcia AP2112K. Tabela \ref{tbl:tab2} porównuje rzeczywiste wartości dziennego poboru prądu z wartościami możliwymi do osiągnięcia poprzez zmianę regulatora napięcia. Tabela \ref{tbl:tab3} porównuje faktyczny osiągnięty czas pracy na baterii z czasem możliwym do osiągnięcia przez zmianę zastosowanego regulatora napięcia.


        \begin{table}[]
            \caption{Porównanie osiągniętego poboru prądu z poborem możliwym do osiągnięcia}
            \centering
            \begin{tabular}{p{6cm}|p{4cm}|p{4cm} }
                    & \textbf{Pobór prądu w trybie uśpienia [mA]} & \textbf{Pobór prądu w trybie aktywnego działania [mA]} \\ \hline
             \textbf{Zmierzone wartości}
                      & 4 & 50 \\
            \hline
            \textbf{Szacowane wartości zakładając wykorzystanie regulatora napięcia AP2112K} &  0,15 & 50  \\
            \end{tabular}
            \label{tbl:tab1}
            \vspace{10mm}
        \end{table}

        \begin{table}[]
            \centering
            \caption{Porównanie osiągniętego dziennego poboru prądu z dziennym poborem możliwym do osiągnięcia}
            \begin{tabular}{p{4cm}|p{3cm}|p{3cm}|p{3cm} }
                    & \textbf{Pobór prądu w trybie uśpienia w ciągu doby [mAh]} & \textbf{Pobór prądu w trybie aktywnego działania w ciągu doby [mAh]} & \textbf{Całkowity pobór prądu w ciągu doby [mAh]} \\ \hline
             \textbf{Zmierzone wartości}
                      & 94,47 & 17,08 & 111,71 \\
            \hline
            \textbf{Szacowane wartości zakładając wykorzystanie regulatora napięcia AP2112K} &  3,54 & 17,08 & 20,63 \\
            \end{tabular}
            \label{tbl:tab2}
            \vspace{10mm}
        \end{table}

        \begin{table}[]
            \caption{Porównanie osiągniętego czasu pracy na baterii z czasem możliwym do osiągnięcia}
            \centering
            \begin{tabular}{p{7,5cm}|p{6,5cm}}
                    & \textbf{Czas pracy na baterii 3300mAh [dni]} \\ \hline
             \textbf{Zmierzone wartości}
                      & 29,53 \\
            \hline
            \textbf{Szacowane wartości zakładając wykorzystanie regulatora napięcia AP2112K} &  159,94  \\
            \end{tabular}
            \label{tbl:tab3}
            \vspace{10mm}
        \end{table}

	\section{Możliwe rozszerzenia}

        W ramach niniejszej pracy stworzony został prototyp rozwiązania. Niektóre planowane funkcjonalności nie zostały zaimplementowane ze względu na ograniczenia czasowe i budżetowe. Pozostawiono jednak możliwość rozbudowy systemu. Poniżej przedstawiono kilka problemów, których system w obecnym stanie nie adresuje, wraz z możliwymi rozwiązaniami.

        \subsection{Mechanizm wyjścia}

            Opisywane rozwiązanie nie obejmuje implementacji mechanizmu opuszczenia strefy chronionej systemem kontroli dostępu. W zależności od potrzeb końcowego użytkownika, możliwe rozwiązanie to montaż dodatkowego czytnika po przeciwnej stronie drzwi i połączenie go z kontrolerem wejścia w przypadku gdy wymagana jest obustronna kontrola dostępu bądź zastosowanie przycisku którego naciśnięcie powoduje zwolnienie zamka w przypadku gdy wymagana jest tylko kontrola wejścia do chronionego obszaru.

        \subsection{Obsługa większej liczby zamków}

            W celu umożliwienia obsługi przez system liczby zamków przekraczającej 1, wystarczająca byłaby modyfikacja podsystemu autoryzacji w taki sposób, aby mógł on obsługiwać równoległe żądania od klientów.

        \subsection{Wygodna konfiguracja parametrów sieci}

            W obecnej implementacji dane dostępu do sieci (nazwa sieci oraz hasło) zostały zagnieżdżone w oprogramowaniu konrolera. Zmniejsza to elastyczność konfiguracji urządzenia, wymagając jego przeprogramowania za każdym razem gdy zmianie ulegnie nazwa lub klucz dostępu do sieci.

            Możliwym rozwiązaniem tego problemu byłaby implementacja trybu konfiguracji. Tryb ten powodowałby przejście kontrolera w tryb Access Point przy zachowaniu dwóch warunków: (1) nastąpiło uruchomienie, a nie wybudzenie z trybu głębokiego uśpienia oraz (2) na określonym wejściu pojawił się stan wysoki. Przejście kontrolera w tryb Access Point umożliwiłoby udostępnienie prostego interfejsu webowego, za pomocą którego administrator systemu mógłby wprowadzić niezbędne do działania dane, takie jak nazwa sieci, hasło, a także adres IP i numer portu serwera autoryzacji. Aby zachować wysoki poziom bezpieczeństwa, komunikacja pomiędzy urządzeniem administratora i kontrolerem powinna odbywać się przy wykorzystaniu protokołu TLS.

        \subsection{Sygnalizacja stanu baterii}

            W obecnym stanie system nie implementuje mechanizmów informowania serwera o swoim stanie energetycznym.
            Sygnalizacja stanu baterii umożliwiłaby administratorowi systemu bieżące monitorowanie wszystkich zamków objętych systemem oraz szybką reakcję w przypadku, gdy baterie wymagałyby wymiany. Modyfikacja ta wymagałaby uzyskania dostępu do danych na temat naładowania baterii przez mikrokontroler oraz przesyłania ich okresowo do serwera, najlepiej w momentach, gdy jest on już wybudzony z powodu wykrycia ruchu w jego otoczeniu.

        \subsection{Obsługa kont użytkowników w podsystemie zarządzania}

            Podsystem zarządzania nie implementuje mechanizmu dostępu do zasobów, co czyni system mniej bezpiecznym. W końcowym produkcie należałoby rozszerzyć go o możliwość tworzenia kont użytkowników i przypisywania im określonych uprawnień do odczytu i zapisu danych.

        \subsection{Zabezpieczenie serwera}

            W celu zapewnienia wysokiego poziomu bezpieczeństwa systemu należałoby zabezpieczyć serwer zaporą sieciową (ang. \textit{firewall}). Jądro systemu Linux zawiera podsystem umożliwiający manipulację pakietów danych. Reguły manipulacji są definiowane przez narzędzie administracyjne \texttt{iptables}. Taka konfiguracja spowodowałaby znaczny wzrost poziomu bezpieczeństwa maszyny serwera.


    \section{Podsumowanie}

        W ramach niniejszej pracy stworzono funkcjonalny prototyp systemu kontroli dostępu do pomieszczeń opartego na bezprzewodowej komunikacji układu zamka i serwera.

        Dzięki wykorzystaniu serwera w celu przeprowadzenia procesu uwierzytelniania system zapewnia większą elastyczność i łatwość zarządzania niż alternatywne systemy wykorzystujące zamki pracujące w sposób autonomiczny. Rozwiązanie cechuje się wygodą montażu, ponieważ nie wymaga przewodów zasilających i komunikacyjnych prowadzonych w ścianach budynków. Przy wdrażaniu rozwiązania nie jest konieczna modyfikacja istniejącej infrastruktury budynku, z wyjątkiem wymiany samych zamków. System nie wymaga żadnych dodatkowych komponentów sprzętowych poza zamkami i serwerem.

        Wydajność energetyczna podsystemu sterowania zamkiem miała zostać osiągnięta przez zarządzanie zasilaniem jego peryferiów oraz kontrolę stanu zasilania mikrokontrolera w celu minimalizacji poboru mocy i maksymalizacji czasu pracy na zasilaniu bateryjnym. Ze względu na wysoki pobór prądu w stanie spoczynku przez regulator napięcia wybranej płytki prototypowej, nie udało się osiągnąć założonej wydajności. Jak opisano w podrozdziale \ref{sec:wydajnosc}, żywotność układu może znacząco wzrosnąć.

        Bezpieczeństwo systemu na wielu poziomach zapewnia wykorzystanie mechanizmów takich jak protokół TLS w warstwie komunikacji pomiędzy zamkiem a serwerem.

        Po stosownych modyfikacjach i rozszerzeniach stworzony system ma szansę efektywnej pracy w odpowiednim środowisku.