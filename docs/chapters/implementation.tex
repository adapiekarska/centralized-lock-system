\chapter{Implementacja}

\section{Uproszczenia}

W ramach niniejszej pracy stworzony został prototyp końcowego rozwiązania. Niektóre z opisywanych w rozdziale \ref{chap:hl-arch} funkcjonalności nie zostały zaimplementowane. Zostawiono jednak możliwość rozbudowy systemu.

Opisywane rozwiązanie nie obejmuje implementacji mechanizmu opuszczenia strefy chronionej systemem kontroli dostępu. W zależności od potrzeb końcowego użytkownika, możliwe rozwiązanie to montaż analogicznego układu czytnika i kontrolera po przeciwnej stronie drzwi w przypadku gdy wymagana jest obustronna kontrola dostępu bądź zastosowanie przycisku którego naciśnięcie powoduje zwolnienie zamka w przypadku gdy wymagana jest tylko kontrola wejścia do chronionego obszaru.

W sytuacjach awaryjnych, jakimi jest brak zasilania bądź brak połączenia z serwerem \textbf{wymyslic co}. Szczegółowy opis tych zagadnień znajduje się \textbf{gdziee?}.

Implementacja prototypu obejmowała stworzenie pojedynczego układu zamka. Ze względu na prototypowy charakter pracy, nie przetestowano działania systemu z większą liczbą zamków. Nie ma jednak powodów by twierdzić, że system nie działałby poprawnie z większą liczbą zamków. Wystarczającym rozszerzeniem byłaby modyfikacja oprogramowania serwera umożliwiająca obsługę kilku klientów jednocześnie. \textbf{czy to pisac?}

\section{Układ sterowania zamkiem}

Prototyp układu sterowania zamkiem powstał w oparciu o platformę ESP32-DevKitC-32D z wbudowanym modułem ESP-WROOM-32D.

\subsection{Platforma}

ESP32-DevKitC jest produkowaną przez firmę Espressif platformą z wbudowanym modułem ESP32-WROOM-32D. Rdzeniem modułu jest układ z rodziny ESP32 (ESP32-D0WD) wyposażony w dwa rdzenie CPU, które mogą być kontrolowane niezależnie od siebie~\cite{esp32-wroom32-ds}. Moduł integruje Bluetooth, Bluetooth Low Energy oraz WiFi, a także szeroki zakres peryferiów: czujniki dotyku, czujniki pola magnetycznego, interfejs karty SD, Ethernet, SPI (ang. Serial Peripheral Interface), UART (ang. Universal Asynchronous Receiver-Transmitter), I\textsuperscript{2}S (ang. Inter-IC Sound) i I\textsuperscript{2}C (ang. Inter-Integrated Circuit)~\cite{esp32-wroom32-ds}.

ESP32 działa w oparciu o system operacyjny freeRTOS.

ESP32 oferuje efektywną i elastyczną technologię zarządzania energią. Istnieje pięć predefiniowanych stanów energetycznych. Dodatkowo umożliwia korzystanie z niskoenergetycznego koprocesora Ultra-Low-Power (ang. ULP co-processor), podczas gdy główne jednostki pozostają w trybie głębokiego uśpienia (ang. Deep-sleep mode).~\cite{esp32-tech-ref-man}


\section{Problemy}
Konfiguracja WiFi?
Sygnalizacja stanu baterii?
Bezpieczeństwo komunikacji
zarządzanie stanami energetycznymi
Enkrypcja flash
mozna cos z technical reference manual