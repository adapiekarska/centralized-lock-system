\chapter{Wstep i cel pracy}
\label{chap:intro}

	\section{Wstęp}
		Bezpieczeństwo przestrzeni wraz z ich użytkownikami jest kwestią kluczową w procesie projektowania infrastruktury obiektów budowlanych, przez co duży nacisk kładzony jest na zapewnienie efektywnego i skutecznego sposobu kontroli dostępu. Idea tworzenia przestrzeni o dostępie ograniczonym towarzyszy rozwojowi cywilizacji od stuleci jednak przez większość tego czasu sposób realizacji tej idei pozostaje niezmienny, oparty na urządzeniu ryglującym z układem zapadkowym rozpoznającym fizyczny klucz. Postępujący rozwój technologiczny pozwala jednak na dokonanie wielu usprawnień w tej dziedzinie, przez co coraz częściej stosowane są systemy oparte na uwierzytelnianiu elektronicznym.

		W elektronicznym systemie kontroli dostępu do pomieszczeń podstawowe założenia pozostają zbieżne z tymi, na których opiera się ich tradycyjny odpowiednik. Dla obu głównym celem jest autoryzacja prób dostępu użytkowników na podstawie kluczy w taki sposób, aby dostęp został przyznany tylko osobie posiadającej powiązany z danym punktem dostępu klucz.
		Często jednak wyznacznikiem skutecznego systemu nie jest tylko sposób autoryzacji, a również koszt i łatwość utrzymania. Elastyczność jaką gwarantują systemy oparte na urządzeniach elektronicznych dają im znaczącą przewagę, jako że rozwiązania tego typu pozwalają na wydajne i spójne zarządzanie, a także szczegółową ewidencję zdarzeń w celu późniejszej analizy.

	\section{Cel i zakres pracy}

		Celem niniejszej pracy jest projekt oraz implementacja sytemu dostępu do pomieszczeń z wykorzystaniem technologii takich jak Wi-Fi oraz RFID (ang. \textit{Radio-frequency identification}), w którym podmiotem odpowiedzialnym za autoryzację prób dostępu jest serwer, a komunikacja pomiędzy układem sterującym zamkiem a podsystemem autoryzacji jest realizowana bezprzewodowo. Może on znaleźć zastosowanie jako łatwy w instalacji i obsłudze, lekki i wydajny system dla małych i średnich obiektów.

		Założenia dotyczące opisywanego systemu są następujące:

		\begin{enumerate}
		    \item Logika uwierzytelniania powinna być zaimplementowana na serwerze. Urządzenie klienckie (zamek) powinno pełnić jedynie rolę pośrednika w tym procesie.
		    \item Komunikacja pomiędzy urządzeniem klienckim (układem zamka) a serwerem powinna odbywać się bezprzewodowo.
		    \item System powinien być wydajny energetycznie i umożliwiać operację zamków na zasilaniu bateryjnym.
		    \item System powinien implementować niezbędne mechanizmy bezpieczeństwa.
		\end{enumerate}

		Pracę nad systemem prowadziły dwie osoby. W ramach tej pracy powstały:
		\begin{itemize}
		    \item Protoyp układu zamka,
		    \item Oprogramowanie serwera uwierzytelniania,
		    \item Aplikacja do zarządzania,
		    \item Baza danych.
		\end{itemize}

		Implementacja prototypu obejmowała stworzenie pojedynczego układu zamka. Ze względu na prototypowy charakter pracy nie przetestowano działania systemu z większą liczbą zamków. Nie ma jednak powodów by twierdzić, że po minimalnych modyfikacjach system nie działałby poprawnie z większą liczbą zamków.

		\textbf{Tutaj podział pracy i obowiązków - TBD}

	\section{Struktura pracy}

		Rozdział \ref{chap:intro} stanowi wstęp do pracy, określa jej cel i strukturę. Rozdział \ref{chap:problem-domain} przedstawia dziedzinę problemu, przywołuje najważniejsze definicje związane z tematem, dokonuje przedstawienia i porównania kilku istniejących rozwiązań, a także opisu podstawowych zagrożeń bezpieczeństwa, z którymi spotkali się autorzy podczas pracy nad systemem. Rozdział \ref{chap:hl-arch} prezentuje projekt rozwiązania. Rozdziały \ref{chap:controller} oraz \ref{chap:server} przedstawiają wynik oraz przebieg procesu implementacyjnego wraz z najciekawszymi problemami oraz przeglądem wykorzystanych technologii odpowiednio po stronie układu zamka oraz serwera. Pracę zamyka rozdział \ref{chap:results}, który opisuje rezultaty pracy nad projektem, przedstawia możliwe rozszerzenia prototypu oraz dokonuje podsumowania, włącznie z oceną użyteczności, bezpieczeństwa oraz spełnienia wymagań.