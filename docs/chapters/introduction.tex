\chapter{Wstep i cel pracy}

	\section{Wstęp}

		Zapewnienie bezpieczeństwa przestrzeni użytkowych i osób z nich korzystających stanowi kluczowy aspekt zarządzania obiektami zarówno publicznymi, jak i prywatnymi. Dzięki zastosowaniu odpowiedniej infrstruktury, bezpieczeństwo osób przebywających na terenie obiektu rośnie, a ryzyko kradzieży lub zniszczenia mienia przez niepowołane osoby spada. Podstawową metodą kontroli dostępu do pomieszczeń jest montaż urządzeń ryglujących z układem zapadkowym rozpoznającym fizyczny klucz. Ze względu na postępujący rozwój technologiczny, w obecnych czasach coraz częściej stosowane są systemy oparte na uwierzytelnianiu elektronicznym.

		Pod względem celu i ogólnych zasad działania elektroniczny system kontroli dostępu do pomieszczeń nie rózni się od swojego tradycyjnego odpowiednika. Głównym celem pozostaje autoryzacja prób dostępu użytkowników na podstawie kluczy w taki sposób, aby dostęp został przyznany tylko osobie posiadającej powiązany z danym punktem dostępu klucz.

		Przewagę systemów opartych na urządzeniach elektronicznych nad systemami czysto mechanicznymi stanowią cechy takie jak łatwość obsługi czy możliwość zdalnego zarządzania oraz zbierania danych i monitorowania prób dostępu w celu późniejszej analizy.

		Celem niniejszej pracy jest projekt oraz implementacja sytemu dostępu do pomieszczeń z wykorzystaniem technologii takich jak Wi-Fi oraz RFID (ang. \textit{Radio-frequency identification}), w którym podmiotem odpowiedzialnym za autoryzację prób dostępu jest serwer, a komunikacja pomiędzy układem sterującym zamkiem a podsystemem autoryzacji jest realizowana bezprzewodowo. Może on znaleźć zastosowanie jako łatwy w instalacji i obsłudze, lekki i wydajny system dla małych i średnich obiektów.

	\section{Cel i zakres pracy}

		Podstawowe założenia dotyczące opisywanego systemu są następujące:

		\begin{enumerate}
		    \item Logika uwierzytelniania powinna być zaimplementowana na serwerze. Urządzenie klienckie (zamek) powinno pełnić jedynie rolę pośrednika w tym procesie.
		    \item Komunikacja pomiędzy urządzeniami klienckimi (zamkami) a serwerem powinna odbywać się bezprzewodowo.
		    \item System powinien być wydajny energetycznie i umożliwiać operację zamków na zasilaniu bateryjnym.
		    \item System powinien implementować niezbędne mechanizmy bezpieczeństwa.
		\end{enumerate}

		Pracę nad systemem prowadziły dwie osoby. W ramach tej pracy powstały:
		\begin{itemize}
		    \item Protoyp układu zamka,
		    \item Oprogramowanie serwera uwierzytelniania,
		    \item Aplikacja do zarządzania,
		    \item Baza danych.
		\end{itemize}

		Implementacja prototypu obejmowała stworzenie pojedynczego układu zamka. Ze względu na prototypowy charakter pracy, nie przetestowano działania systemu z większą liczbą zamków. Nie ma jednak powodów by twierdzić, że po minimalnych modyfikacjach system nie działałby poprawnie z większą liczbą zamków.

		\textbf{Tutaj podział pracy i obowiązków - TBD}

	\section{Struktura pracy}

		Rozdział \ref{chap:problem-domain} pokrótce przedstawia dziedzinę problemu, przywołuje najważniejsze definicje związane z tematem oraz ogólny opis działania systemów kontroli dostępu. Opisuje też podstawowe zagrożenia bezpieczeństwa, z którymi spotkali się autorzy podczas pracy nad systemem, a także dokonuje przedstawienia i porównania kilku istniejących na rynku rozwiązań. Rozdział \ref{chap:hl-arch} prezentuje projekt rozwiązania. Rozdział \ref{chap:implementation} przedstawia proces implementacji tego projektu wraz z prezentacją najciekawszych problemów implementacyjnych oraz wykorzystanych technologii. Ze względu na to, iż w ramach pracy zaimplementowany został prototyp rozwiązania, rozdział ten prezentuje również możliwe modyfikacje i rozszerzenia tego prototypu. Rozdział \ref{chap:results} opisuje rezultaty pracy nad projektem. Pracę zamyka rozdział \ref{chap:conclusions}, który dokonuje podsumowania.