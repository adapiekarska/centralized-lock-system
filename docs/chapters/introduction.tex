\newchapter{Wstep i cel pracy}{Adrianna Piekarska}
\label{chap:intro}

	\section{Wstęp}
		Zapewnienie bezpieczeństwa przestrzeni oraz ich użytkowników stanowi istotny etap procesu projektowania infrastruktury obiektów budowlanych. Jednym ze środków podejmowanych w celu finalizacji tego etapu jest implementacja efektywnego i skutecznego systemu kontroli dostępu do chronionych stref.

		Idea tworzenia przestrzeni o ograniczonym dostępie towarzyszy rozwojowi cywilizacji od stuleci. Przez znaczną część tego czasu sposób jej realizacji pozostawał niezmienny i opierał się na wykorzystaniu prostego urządzenia ryglującego wraz z układem zapadkowym zdolnym do rozpoznawania fizycznych kluczy. Jednakże postęp technologiczny dotyczący wszystkich sfer ludzkiego życia umożliwił wprowadzenie szeregu usprawnień również w tej dziedzinie, przynosząc popularność systemom kontroli dostępu opartym na uwierzytelnianiu elektronicznym.

		Głównym celem zarówno elektronicznego systemu, jak i jego tradycyjnego odpowiednika, jest autoryzacja prób dostępu użytkowników na podstawie kluczy w taki sposób, aby dostęp został przyznany tylko użytkownikowi posiadającemu powiązany z danym punktem dostępu klucz.

		Wyznacznikiem skuteczności i użyteczności systemu kontroli dostępu jest nie tylko metoda czy poprawność autoryzacji, lecz także koszt i łatwość instalacji oraz utrzymania. Znaczącą przewagę systemów elektronicznych stanowi ich elastyczność. Rozwiązania tego typu zapewniają możliwość łatwego i wydajnego zarządzania oraz szczegółowej ewidencji zdarzeń celem analizy ruchu na terenie obiektu.

	\section{Cel i zakres pracy}

		Celem niniejszej pracy jest projekt oraz implementacja sytemu kontroli dostępu do pomieszczeń z wykorzystaniem technologii takich jak Wi-Fi i RFID (ang. \textit{Radio-frequency identification}), w którym podmiotem odpowiedzialnym za autoryzację prób dostępu jest serwer, a komunikacja pomiędzy urządzeniem klienckim (układem sterującym zamkiem) a oprogramowaniem autoryzacyjnym jest realizowana bezprzewodowo.

		Założenia dotyczące systemu są następujące:

		\begin{enumerate}
		    \item Logika uwierzytelniania powinna być przeprowadzana na serwerze. Urządzenie klienckie powinno pełnić jedynie rolę pośrednika w tym procesie.
		    \item Komunikacja pomiędzy urządzeniem klienckim a serwerem powinna odbywać się bezprzewodowo.
		    \item System powinien być wydajny energetycznie i umożliwiać operację układów zamka na zasilaniu bateryjnym.
		    \item System powinien implementować niezbędne mechanizmy bezpieczeństwa.
		\end{enumerate}

		Pracę nad systemem prowadziły dwie osoby. W ramach tej pracy powstały:

		\begin{itemize}
		    \item Protoyp układu zamka,
		    \item Oprogramowanie serwera uwierzytelniania,
		    \item Oprogramowanie służące do zarządzania,
		    \item Baza danych.
		\end{itemize}

		Implementacja prototypu obejmowała stworzenie pojedynczego układu zamka. Ze względu na prototypowy charakter pracy nie przetestowano działania systemu z większą liczbą zamków. Nie ma jednak powodów by twierdzić, że po minimalnych modyfikacjach system nie działałby poprawnie z większą liczbą zamków.


	\section{Struktura pracy}

		Rozdział \ref{chap:intro} stanowi wstęp do pracy, określa jej cel i strukturę. Rozdział \ref{chap:problem-domain} przedstawia dziedzinę problemu, przywołuje najważniejsze definicje związane z tematem, dokonuje przedstawienia i porównania kilku istniejących rozwiązań, a także opisu podstawowych zagrożeń bezpieczeństwa, z którymi spotkali się autorzy podczas pracy nad systemem. Rozdział \ref{chap:hl-arch} prezentuje projekt rozwiązania. Rozdziały \ref{chap:controller} oraz \ref{chap:server} przedstawiają wynik oraz przebieg procesu implementacyjnego wraz z najciekawszymi problemami oraz przeglądem wykorzystanych technologii odpowiednio po stronie układu zamka oraz serwera. Pracę zamyka rozdział \ref{chap:results}, który zawiera podział prac pomiędzy członków zespołu, opisuje rezultaty pracy nad projektem, przedstawia przeprowadzone testy oraz możliwe rozszerzenia prototypu, a także dokonuje podsumowania, włącznie z oceną użyteczności, bezpieczeństwa oraz spełnienia wymagań. 