\chapter{Wstep i cel pracy}

Zapewnienie bezpieczeństwa przestrzeni użytkowych i osób z nich korzystających stanowi kluczowy aspekt zarządzania obiektami zarówno publicznymi, jak i prywatnymi. Dzięki zastosowaniu odpowiedniej infrstruktury, bezpieczeństwo osób przebywających na terenie obiektu rośnie, a ryzyko kradzieży lub zniszczenia mienia przez niepowołane osoby spada. Podstawową metodą kontroli dostępu do pomieszczeń są metody mechaniczne wykorzystujące jedynie fizyczne zabezpieczenia. Ze względu na postępujący rozwój technologiczny, w obecnych czasach coraz częściej stosowane są systemy oparte na uwierzytelnianiu elektronicznym.

Pod względem celu i ogólnych zasad działania, elektroniczny system kontroli dostępu do pomieszczeń nie rózni się od swojego tradycyjnego odpowiednika. Głównym celem pozostaje autoryzacja prób dostępu użytkowników na podstawie kluczy w taki sposób, aby dostęp został przyznany tylko osobie posiadającej powiązany z danym punktem dostępu klucz.

Przewagę systemów opartych na urządzeniach elektronicznych nad systemami czysto mechanicznymi stanowią cechy takie jak łatwość obsługi czy możliwość zdalnego zarządzania oraz zbierania danych i monitorowania prób dostępu w celu późniejszej analizy.

Celem niniejszej pracy jest projekt oraz implementacja sytemu dostępu do pomieszczeń z wykorzystaniem technologii takich jak WiFi oraz RFID (ang. \textit{Radio-frequency identification}), w którym podmiotem odpowiedzialnym za autoryzację prób dostępu jest serwer, a komunikacja pomiędzy podsystemem sterowania zamkiem a podsystemem autoryzacji jest realizowana bezprzewodowo. Może on znaleźć zastosowanie jako łatwy w instalacji i obsłudze, lekki i wydajny system dla małych i średnich obiektów.

Podstawowe założenia dotyczące opisywanego systemu są następujące:

\begin{enumerate}
    \item Logika uwierzytelniania powinna być zaimplementowana na serwerze. Urządzenie klienckie (zamek) powinno pełnić jedynie rolę pośrednika w tym procesie.
    \item Komunikacja pomiędzy urządzeniami klienckimi (zamkami) a serwerem powinna odbywać się bezprzewodowo.
    \item System powinien być wydajny energetycznie i umożliwiać operację zamków na zasilaniu bateryjnym.
    \item System powinien implementować niezbędne mechanizmy bezpieczeństwa.
\end{enumerate}

Dzięki wykorzystaniu zdalnego serwera do przeprowadzenia procesu uwierzytelniania system zapewnia większą elastyczność i łatwość zarządzania niż alternatywne systemy wykorzystujące zamki pracujące w sposób autonomiczny. Informacje o uprawnieniach przechowywane są w centralnej bazy danych, znajdującej się na serwerze, którą można w prosty sposób zarządzać z poziomu aplikacji internetowej.

Rozwiązanie cechuje się wygodą montażu, ponieważ nie wymaga przewodów zasilających i komunikacyjnych prowadzonych w ścianach budynków. Przy wdrażaniu rozwiązania nie jest konieczna modyfikacja istniejącej infrastruktury budynku, z wyjątkiem wymiany samych zamków. System nie wymaga żadnych dodatkowych komponentów sprzętowych poza zamkami i serwerem. Do poprawnego działania systemu potrzebna jest sieć WiFi. Założono, że wykorzystana sieć nie musi być bezpieczna.

Wydajność energetyczna podsystemu sterowania zamkiem została osiągnięta przez zarządzanie zasilaniem jego peryferiów oraz kontrolę stanu zasilania mikrokontrolera w celu minimalizacji poboru mocy i maksymalizacji czasu pracy na zasilaniu bateryjnym.

Bezpieczeństwo systemu na wielu poziomach zapewnia wykorzystanie mechanizmów takich jak TLS (ang. \textit{Transport Layer Security}) w warstwie komunikacji pomiędzy zamkiem a serwerem czy szyfrowanie pamięci Flash w warstwie operacji na danych w mikroprocesorze w układzie zamka.

\section{Zakres pracy}
Pracę nad systemem prowadziły dwie osoby. W ramach tej pracy powstały:
\begin{itemize}
    \item Protoyp układu zamka,
    \item Implementacja logiki uwierzytelniania,
    \item Implementacja prostej aplikacji do zarządzania. \textbf{???????}
\end{itemize}

Praca nie obejmuje zdefiniowania zachowania systemu w przypadkach awarii. Zachowanie to można zdefiniować w dowolny sposób i rozszerzyć prototyp o mechanizmy niezbędne do jego wsparcia.

\textbf{tutaj cos o podziale pracy}

\section{Struktura pracy}

Rozdział \ref{chap:problem-domain} pokrótce przedstawia dziedzinę problemu, przywołuje najważniejsze definicje związane z tematem oraz ogólny opis działania systemów kontroli dostępu. Dokonuje także przedstawienia oraz porównania kilku istniejących na rynku rozwiązań.

Rozdział \ref{chap:hl-arch} prezentuje projekt rozwiązania.

Rozdział \ref{chap:implementation} przedstawia proces implementacji systemu wraz z prezentacją najciekawszych problemów implementacyjnych oraz wykorzystanych technologii. W ramach pracy zaimplementowany został prototyp rozwiązania. Rozdział prezentuje również możliwe modyfikacje i rozszerzenia tego prototypu.