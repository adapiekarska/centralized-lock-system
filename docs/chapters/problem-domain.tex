\chapter{Dziedzina problemu}
\label{chap:problem-domain}
	\textbf{TBD: dodac sekcję z najważniejszymi pojęciami}

	Niniejszy rozdział definiuje dziedzinę problemu, przedstawia i porównuje kilka istniejących rozwiązań oraz nakreśla podstawowe zagadnienia związane z bezpieczeństwem tego typu systemów.

	\section{Kontrola dostępu}

		Kontrola dostępu definiowana jest w \cite{pkn2002} jako środki mające na celu zapewnienie, że do zasobów systemu przetwarzania danych mogą mieć dostęp tylko uprawnione jednostki w uprawniony sposób.

		British Security Industry Association wyodrębnia kilka komponentów składających się na system kontroli dostępu~\cite{bsia2016}. Poniżej przedstawiono najważniejsze z nich.

		Poświadczenie tożsamości (ang. \textit{credentials}) to fizyczny lub materialny obiekt, element wiedzy lub cecha biometryczna umożliwiająca uzyskanie dostępu do kontrolowanej strefy. Najczęściej jako poświadczenie tożsamości stosuje się kody, np. PIN (ang. \textit{Personal Identification Number}, osobisty numer identyfikacyjny), tokeny (karty, urządzenia mobilne itp.) oraz dane biometryczne.

		British Security Industry Association terminem ''czytniki'' (ang. \textit{readers}) nazywa urządzenia odpowiedzialne za kontrolę dostępu. Dla uproszczenia nomenklatura ta została zachowana w niniejszej sekcji. W innych częściach niniejszej pracy termin ''czytnik'' używany jest w znaczeniu urządzenia odpowiedzialnego wyłącznie za odczyt danych z nośnika.

		W większości przypadków tylko wejście podlega kontroli. Aby możliwa była również kontrola wyjścia z chronionego terenu, potrzebny jest drugi czytnik umieszczony po drugiej stronie drzwi. Jeżeli obustronna kontrola nie jest wymagana, stosuje się zazwyczaj przycisk umożliwiający otworzenie zamka od środka.

		Urządzenia wyjściowe (ang. \textit{egress devices}) umożliwiają użytkownikowi opusczenie strefy chronionej od wewnątrz. Jako urządzenia wyjściowe najczęściej używa się przełączników, czujników ruchu lub czytników. Według British Security Industry Association urządzenia wyjściowe można podzielić je na zwykłe oraz awaryjne (ang. \textit{emergency egress}), przy czym, ze względu na krytyczne znaczenie w wypadku awarii, działanie tych drugich nie powinno zależeć od komponentów systemu. Jako urządzenie awaryjne często stosuje się tzw. \textit{break glass device}, którego uaktywnienie powoduje odcięcie zasilania w zamku, a tym samym wstrzymanie kontroli dostępu w danym punkcie. Dostęp uzyskany za pomocą tego urządzenia powinien wygenerować stosowne powiadomienie bądź alarm.

	\section{Przegląd istniejących rozwiązań}

		Rozwinięta technologia dostarcza wiele możliwości w kwestii implementacji systemów kontroli dostępu. Nie wszyscy zarządcy obiektów decydują się jednak na wykorzystanie nowoczesnych systemów. Prowadzi to do znacznej różnorodności rozwiązań, z których każde posiada zalety i wady właściwe dla wykorzystanych metod. Poniżej przedstawiona została zaproponowana w \cite{access-system-survey} klasyfikacja systemów kontroli dostępu ze względu na metodę uwierzytelniania:
		\begin{enumerate}[label=\Alph*.]
			\item System zamków mechanicznych

				W tego typu systemie zamek zawiera układ mechaniczny otwierany przez fizyczny klucz. Jest to wciąż najczęściej stosowana metoda ochrony przestrzeni mieszkalnych, mimo że złamanie takiego zabezpieczenia wymaga jedynie odpowiednich umiejętności manualnych i wykorzystanych narzędzi.

			\item System z uwierzytelnieniem przez PIN

				Dostęp do chronionych pomieszczeń w tego rodzaju systemie przydzielany jest na podstawie kodu wprowadzanego za pomocą klawiatury numerycznej podłączonej do mikrokontrolera. Rozwiązanie to cechuje się brakiem konieczności posiadania kosztownego w produkcji unikalnego, fizycznego klucza, który z natury narażony jest na kradzież lub zagubienie~\cite{keypad-access-system}. Wadą tego rozwiazania jest ograniczona możliwość zmiany kodu dostępu i podatność na jego wydobycie z pamięci programowalnej.

			\item System z uwierzytelnieniem przez RFID

				Rolę nośnika klucza w tego typu rozwiązaniu pełni identyfikator RFID. W przypadku dostatecznego zbliżenia nośnika do anteny umieszczonej w drzwiach lub ich bezpośrednim sąsiedztwie, następuje odczyt i weryfikacja klucza. Sytemy tego typu umożliwiają śledzenie ruchu użytkowników systemu w czasie rzeczywistym oraz monitorowanie wyposażenia objętego kontrolą dostępu przez umieszczenie w nim identyfikatorów RFID, jak zostało to opisane w \cite{rfid-access-system-for-university}. Ze względu na wykorzystanie nieszyfrowanej transmisji bezprzewodowej, przesyłany klucz jest łatwy do przechwycenia.

			\item System z uwierzytelnieniem przez dane biometryczne

				Opierając uwierzytelnienie o konwersję unikalnych cech fizycznych użytkownika do postaci numerycznej można całkowicie wyeliminować konieczność istnienia fizycznych kluczy i haseł. Zaletą takiego systemu jest brak możliwości zgubienia lub kradzieży klucza czy zapomnienia hasła. Potencjalnie, cecha ta podnosi poziom bezpieczeństwa obiektu objętego tego typu kontrolą dostępu, jednak rozwiązanie posiada też wady. Wśród przezentowanych w \cite{biometric-system-vulnerabilities} zagrożeń wyróżnić można szczególnie brak tajności danych biometrycznych, co sprawia, że możliwa jest duplikacja danej cechy. Sama cecha biometryczna jest podatna na uszkodzenia mechaniczne, a po utracie nie można jej zastąpić.

			\item System z uwierzytelnieniem przez OTP (ang. \textit{One Time Password}, hasło jednorazowe)

				Uwierzytenienie w systemie tego rodzaju wymaga od użytkownika wprowadzenia jednorazowego hasła. Hasło to jest generowane przez serwer i zapamiętywane w systemie, a do użytkownika wysyłane w wiadomości tekstowej. Podczas dokonywania próby dostępu użytkownik podaje otrzymane hasło, które zostaje porównane z hasłem zapamiętanym w systemie. Po dokonaniu porównania hasło jest usuwane z systemu. Główną zaletą tego rozwiązania jest zmniejszona podatność na ataki powtórzeniowe, w których atakujący przechwytuje dane wykorzystane do autoryzacji i używa ich ponownie w celu uzyskania dostępu. Problemem jest jednak konieczność przebywania użytkownikownika w zasięgu sieci GSM w celu przekazania hasła za pomocą wiadomości SMS.

			\item System oparty na kryptograficznym bezpieczeństwie danych

				Bezpieczeństwo tego rozwiązania bazuje na zachowaniu poufności danych. Kontrola dostępu polega na porównaniu hasła zaprezentowanego przez użytkownika z hasłem zapisanym w pamięci urządzenia w zaszyfrowanej postaci. Hasło jest ustanawiane przez użytkownika w procesie nadawania uprawnień dostępu. Zmniejsza to ryzyko naruszenia poufności hasła i zapewnia, że nawet administrator systemu nie ma możliwości uzyskania dostępu do poufnych danych uzytkowników. Słabą stroną metody jest konieczność oparcia bezpieczeństwa na bezpieczeństwie klucza wykorzystanego w procesie szyfrowania.

			\item System oparty o komunikację bezprzewodową

				W systemie tego typu przepływ danych odbywa się za pośrednictwem sieci bezprzewodowej. W celu autoryzacji wykorzystywane jest urządzenie mobilne połączone z siecią. W przeciwieństwie do pozostałych wymienionych rozwiązań, rozwiązanie zaproponowane w \cite{cryptographic-iot-access-system} nie zakłada bezpośredniej komunikacji użytkownika z zamkiem. Urządzenie mobilne użytkownika generuje zapytanie do serwera, który dokonuje kontroli uprawnień i w przypadku pozytywnej ich weryfikacji, przekazuje decyzję o sukcesie do układu zamka.


		\end{enumerate}

		Każde z zaprezentowanych rozwiązań ma zarówno zalety jak i wady, które mogą wynikać z założeń, ograniczeń technologicznych lub charakterystyki środowiska wdrożenia. Rzeczywiste systemy zazwyczaj można zaklasyfikować do więcej niż jednej kategorii, ponieważ projektowanie skutecznych rozwiązań wymaga łączenia różnych metod w celu maksymalizacji korzyści płynących z ich zastosowania. Rzeczywiste scenariusze rzadko dają się objąć ścisłą klasyfikacją i wymagają adaptacji wzorców w celu uzyskania najlepszych rezultatów.

	\section{Bezpieczeństwo}

		Bezpieczeństwo jest jednym z głównych wyznaczników jakości systemu kontroli dostępu. Wielu producentów systemów elektronicznych nie podejmuje kroków w kierunku zapewnienienia odpowiedniego poziomu bezpieczeństwa swoich produktów, co często prowadzi do narażenia konsumentów na straty. Dlatego należy poświęcić odpowiednią uwagę zagadnieniom związanym z potencjalymi zagrożeniami i możliwymi słabymi punktami architektury. Rozproszone systemy elektroniczne są sczególnym wyzwaniem z punktu widzenia bezpieczeństwa. Poniżej przedstawiono wybrane punkty zaproponowanej w \cite{iot-vulnerabilities} klasyfikacji potencjalnych podatności w systemach IoT (ang. \textit{Internet of Things}, Internet Rzeczy):
		
		\begin{enumerate}[label=\Alph*.]
			\item Bezpieczeństwo fizyczne

				Urządzenia stanowiące część systemu często umieszczane są w miejscach nieposiadających odpowiedniej fizycznej ochrony. Stwarza to ryzyko zakłócenia pracy urządzenia i wydobycia wrażliwych danych, włącznie z oprogramowaniem i kluczami kryptograficznymi. Jak przedstawia \cite{iot-hardware-attack}, producenci często skupiają się na zagrożeniach związanych z sieciami, nie uwzględniając możliwości ataku o charakterze sprzętowym. Brak poświęcenia uwagi bezpieczeństwu sprzętowemu umożliwił autorom wspomnianej pracy zmianę konfiguracji procesu uruchomienia przez port szeregowy i w efekcie zdobycie uprawnień administratora w systemie operacyjnym działającym na urządzeniu będącym przedmiotem ataku.

			\item Dostępność

				Węzły systemu dysponujące ograniczonymi zasobami energii podatne są na tzw. atak wyczerpania zasobów (ang. \textit{resource depletion attack}). Jak sugeruje nazwa, celem tego ataku jest skończony zasób o krytycznym z punktu widzenia funkcjonowania danego urządzenia znaczeniu. W przypadku opisywanego ataku zasobem tym jest źródło zasilania w postaci baterii o ograniczonej pojemności. W przeciwieństwie do standardowych ataków DoS (ang. \textit{Denial of Service}), ataki wyczerpania zasobów są rozłożone w czasie i dążą do osiągnięcia długotrwałego efektu odmowy usługi~\cite{iot-rd-attack}.

			\item Poufność danych

				W systemach krytycznych działających w oparciu o komunikację bezprzewodową bezpieczeństwo danych stanowi punkt wyjścia dla rozważań o bezpieczeństwie całego systemu. Szyfrowanie jest efektywnym sposobem ochrony danych przed nieautoryzowanym odczytem, ma jednak swoje ograniczenia. Algorytmy kryptograficzne często są wymagające obliczeniowo, co stanowi problem zwłaszcza w przypadku platform sprzętowych o ograniczonych możliwościach. Ponadto rozwiązania kryptograficzne są bezpieczne tylko w takim stopniu, w jakim bezpieczny jest klucz szyfrujący. Stawia to przed urządzeniem konieczność bezpiecznego przechowywania klucza, co biorąc pod uwagę podatnośc na ataki sprzętowe stanowi istotny problem.

		\end{enumerate}

		Podsumowując, bezpieczeństwo stanowi nieodzowny problem z którym muszą zmierzyć się projektanci systemów kontroli dostępu. Nie jest możliwe stworzenie bezpiecznego systemu bez zapewnienia stosownych mechanizmów ochrony wszystkich komponentów z których się składa. Aby z powodzeniem uchronić system przed atakami należy z uwagą przeanalizować potrzeby, wymagania oraz potencjalne zagrożenia mogące występować w środowisku wdrożenia. 