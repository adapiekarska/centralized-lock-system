\chapter{Dziedzina problemu}
\label{chap:problem-domain}

	Niniejszy rozdział definiuje dziedzinę problemu, przedstawia i porównuje kilka istniejących rozwiązań oraz nakreśla podstawowe zagadnienia związane z bezpieczeństwem tego typu systemów.

	\section{Kontrola dostępu}

		Kontrola dostępu definiowana jest w \cite{pkn2002} jako środki mające na celu zapewnienie, że do zasobów systemu przetwarzania danych mogą mieć dostęp tylko uprawnione jednostki w uprawniony sposób.

		British Security Industry Association wyodrębnia kilka komponentów składających się na system kontroli dostępu~\cite{bsia2016}. Poniżej przedstawiono najważniejsze z nich.

		Poświadczenie tożsamości (ang. \textit{credentials}) to fizyczny lub materialny obiekt, element wiedzy lub cecha biometryczna umożliwiająca uzyskanie dostępu do kontrolowanej strefy. Najczęściej jako poświadczenie tożsamości stosuje się kody, np. PIN (ang. \textit{Personal Identification Number}, osobisty numer identyfikacyjny), tokeny (karty, urządzenia mobilne itp.) oraz dane biometryczne.

		British Security Industry Association terminem ''czytniki'' (ang. \textit{readers}) nazywa urządzenia odpowiedzialne za kontrolę dostępu. Dla uproszczenia nomenklatura ta została zachowana w niniejszej sekcji. W innych częściach niniejszej pracy termin ''czytnik'' używany jest w znaczeniu urządzenia odpowiedzialnego wyłącznie za odczyt danych z nośnika.

		W większości przypadków tylko wejście podlega kontroli. Aby możliwa była również kontrola wyjścia z chronionego terenu, potrzebny jest drugi czytnik umieszczony po drugiej stronie drzwi. Jeżeli obustronna kontrola nie jest wymagana, stosuje się zazwyczaj przycisk umożliwiający otworzenie zamka od środka.

		Urządzenia wyjściowe (ang. \textit{egress devices}) umożliwiają użytkownikowi opusczenie strefy chronionej od wewnątrz. Jako urządzenia wyjściowe najczęściej używa się przełączników, czujników ruchu lub czytników. Według British Security Industry Association urządzenia wyjściowe można podzielić je na zwykłe oraz awaryjne (ang. \textit{emergency egress}), przy czym, ze względu na krytyczne znaczenie w wypadku awarii, działanie tych drugich nie powinno zależeć od komponentów systemu (kontrolera systemu, oprogramowania itp.). Jako urządzenie awaryjne często stosuje się tzw. \textit{break glass device}, którego uaktywnienie powoduje odcięcie zasilania w zamku, a tym samym wstrzymanie kontroli dostępu w danym punkcie. Dostęp uzyskany za pomocą tego urządzenia powinien wygenerować stosowne powiadomienie bądź alarm.

		W przypadku, gdy system kontroli dostępu nie funkcjonuje odpowiednio (np. z powodu braku zasilania), stosuje się tzw. \textit{break glass device}.

		W zależności od potrzeb, oprogramowanie w systemie może być samodzielnym programem zainstalowanym na komputerze osobistym bądź złożonym i bezpiecznym oprogramowaniem zainstalowanym na serwerze. Często opiera się na rozwiązaniach webowych lub mobilnych, umożliwiając dostęp z dowolnego urządzenia

	\section{Przegląd istniejących rozwiązań}

		Obecny w dzisiejszych czasach postęp technologiczny zapewnia wiele możliwości w kwestii implementacji systemów kontroli dostępu. Prowadzi to do różnorodności rozwiązań, z których każde posiada właściwe dla wykorzystanych metod wady i zalety. Poniżej przedstawiona została zaproponowana w \cite{access-system-survey} klasyfikacja systemów kontroli dostępu ze względu na metodę uwierzytelniania:
		\begin{enumerate}[label=\Alph*.]
			\item System zamków mechanicznych

				W tego typu systemie zamek zawiera układ mechaniczny otwierany przez fizyczny klucz. Jest to wciąż najczęściej stosowana metoda ochrony przestrzeni mieszkalnych, mimo że złamanie takiego zabezpieczenia wymaga jedynie odpowiednich umiejętności manualnych i wykorzystanych narzędzi.

			\item System z uwierzytelnieniem przez PIN

				Dostęp do chronionych pomieszczeń w tego rodzaju systemie przydzielany jest na podstawie kodu wprowadzanego za pomocą klawiatury numerycznej podłączonej do mikrokontrolera. Rozwiązanie to cechuje się brakiem konieczności posiadania kosztownego w produkcji unikalnego, fizycznego klucza, który z natury narażony jest na kradzież lub zagubienie~\cite{keypad-access-system}. Wadą tego rozwiazania jest ograniczona możliwość zmiany kodu dostępu i podatność na jego wydobycie z pamięci programowalnej.

			\item System z uwierzytelnieniem przez RFID

				Rolę nośnika klucza w tego typu rozwiązaniu pełni identyfikator RFID. W przypadku dostatecznego zbliżenia nośnika do anteny umieszczonej w drzwiach lub ich bezpośrednim sąsiedztwie, następuje odczyt i weryfikacja klucza. Sytemy tego typu umożliwiają śledzenie ruchu użytkowników systemu w czasie rzeczywistym oraz monitorowanie wyposażenia objętego kontrolą dostępu przez umieszczenie w nim identyfikatorów RFID, jak zostało to opisane w \cite{rfid-access-system-for-university}. Ze względu na wykorzystanie nieszyfrowanej transmisji bezprzewodowej, przesyłany klucz jest łatwy do przechwycenia.

			\item System z uwierzytelnieniem przez dane biometryczne

				Opierając uwierzytelnienie o konwersję unikalnych cech fizycznych użytkownika do postaci numerycznej można całkowicie wyeliminować konieczność istnienia fizycznych kluczy i haseł. Zaletą takiego systemu jest brak możliwości zgubienia lub kradzieży klucza czy zapomnienia hasła. Potencjalnie, cecha ta podnosi poziom bezpieczeństwa obiektu objętego tego typu kontrolą dostępu, jednak rozwiązanie posiada też wady. Wśród przezentowanych w \cite{biometric-system-vulnerabilities} zagrożeń wyróżnić można szczególnie brak tajności danych biometrycznych, co sprawia, że możliwa jest duplikacja danej cechy. Sama cecha biometryczna jest podatna na uszkodzenia mechaniczne, a po utracie nie można jej zastąpić.

			\item System z uwierzytelnieniem przez OTP (ang. \textit{One Time Password}, hasło jednorazowe)

				Uwierzytenienie w systemie tego rodzaju wymaga od użytkownika wprowadzenia jednorazowego hasła. Hasło to jest generowane przez serwer i zapamiętywane w systemie, a do użytkownika wysyłane w wiadomości tekstowej. Podczas dokonywania próby dostępu użytkownik podaje otrzymane hasło, które zostaje porównane z hasłem zapamiętanym w systemie. Po dokonaniu porównania jest ono usuwane z systemu. Główną zaletą tego rozwiązania jest znacząco zmniejszona podatność na ataki powtórzeniowe, w których atakujący przechwytuje dane wykorzystane do autoryzacji i używa ich ponownie w celu uzyskania dostępu. Problemem jest jednak konieczność dostępności użytkownikownika w sieci GSM w celu przekazania hasła za pomocą wiadomości SMS.

			\item System oparty na kryptograficznym bezpieczeństwie danych

				Bezpieczeństwo tego rozwiązania bazuje na kluczu dostępu znanym wyłącznie przez użytkownika systemu. W pamięci układu zamka zapisywana jest odpowiednia sekwencja, która po transformacji kryptograficznej staje się hasłem dostępu (w tej postaci jest ono przekazywane użytkownikowi). W ten sposób hasło znane jest tylko użytkownikowi co zapewnia, że nawet osoby odpowiedzialne za utrzymanie systemu nie uzyskają dostępu przez pozyskanie hasła. Słabą stroną metody jest oparcie bezpieczeństwa na kluczu kryptograficznym który musi być przechowywany w bezpieczny sposób. \textbf{Nie rozumiem o co tu chodzi.}

			\item System oparty o komunikację bezprzewodową
				\textbf{TODO: review documentation considering author names}
				W systemie tego typu przepływ danych odbywa się za pośrednictwem sieci bezprzewodowej. W celu autoryzacji wykorzystywane jest urządzenie mobilne połączone z siecią. W przeciwieństwie do pozostałych wymienionych rozwiązań, rozwiązanie zaproponowane w \cite{cryptographic-iot-access-system} nie zakłada bezpośredniej komunikacji użytkownika z zamkiem. Urządzenie mobilne użytkownika generuje zapytanie do serwera, który dokonuje kontroli uprawnień i w przypadku pozytywnej ich weryfikacji, przekazuje decyzję o sukcesie do układu zamka.

		\end{enumerate}

		\textbf{Tu może jakieś podsumowanie tych różnych rodzajów jeszcze jakby się dało}

	\section{Bezpieczeństwo}
		\textbf{Przegląd zagrożeń związanych z bezpieczeństwem - TBD}



		Rynek systemów IoT nastawiony jest przede wszystkim na zysk. W efekcie wielu producentów w obliczu ograniczeń czasowych i budżetowych nie podejmuje prac związanych z zapewnieniem odpowiedniego poziomu bezpieczeństwa. Stosowanie takiej praktyki może jednak mieć poważne konsekwencje, jako że naraża użytkowników systemu na straty

		W projekcie systemu realizującego funkcjonalność o znaczeniu krytycznym, taką jak kontrola dostępu, jego bezpieczeństwo jest głównym wyznacznikiem jakości. Z tego powodu należy poświęcić odpowiednią uwagę zagadnieniom związanym z potencjalymi zagrożeniami i możliwymi słabymi punktami architektury. Systemy realizujące ideę Internetu Rzeczy są sczególnym wyzwaniem z punktu widzenia bezpieczeństwa. Poniżej przedstawiono wybrane punkty zaproponowanej w \cite{iot-vulnerabilities} klasyfikacji potencjalnych zagrożeń w systemach IoT:
		\begin{enumerate}[label=\Alph*.]
			\item Bezpieczeństwo fizyczne

				Urządzenia tworzące system częto umieszczane są w miejscach bez odpowiedniej fizycznej ochrony. Stwarza to możliwość zakłócenia pracy urządzenia czy wydobycia wrażliwych danych włącznie z kodem i kluczami kryptograficznymi. Jak przedstawia \cite{iot-hardware-attack}, producenci często skupiają się na zagrożeniach sieciowych bez uwzględnienia możliwości ataku sprzętowego. Pozwoliło to autorom pracy uzyskać niskopoziomowy dostęp konfigurując proces uruchomienia przez port szeregowy i w efekcie zdobyć uprawnienia administratora w systemie operacyjnym działającym na urządzeniu.

			\item Dostępność

				Wezły systemu dysponujące ograniczonym zasobem energii podatne są na tak zwany atak wyczerpania zasobów (ang. \textit{resource depletion attack}). Jak sugeruje nazwa, celem tego ataku jest skończony zasób, krytyczny dla funkcjonowania danego urządzenia. W tym przypadku owym zasobem jest źródło zasilania w postaci baterii o ograniczonej pojemności. W przeciwieństwie do standardowych ataków DoS (ang. \textit{Denial of Service}, Odmowa Usługi), ataki wyczerpania zasobów są rozłożone w czasie i dążą do osiągnięcia długotrwałej odmowy usługi~\cite{iot-rd-attack}.

			\item Poufność danych

				Bezpieczeństwo danych, szczególnie w systemach krytycznych oraz wykorzystujących komunikację bezprzewodową, jest najczęściej punktem wyjścia dla rozważań o bezpieczeństwie całego systemu. Szyfrowanie jest efektywnym sposobem ochrony danych przed nieautoryzowanym odczytem. Ma ono jednak swoje ograniczenia. Algorytmy kryptograficzne są wymagające obliczeniowo co ma spore znaczenie w przypadku platform sprzętowych o ograniczonych możliwościach. Ponadto rozwiązania kryptograficzne są tak bezpieczne, jak bezpieczny jest klucz szyfrujący, co stawia przed urządzeniem wymaganie bezpiecznego przehowywania klucza jest zadaniem nietrywialnym biorąc pod uwagę fakt, że urządzenie podatne jest na sprzętowe ataki.

		\end{enumerate}



		% W mniej wymagających systemach często stosuje się rozwiązania oparte na architekturze rozproszonej. W rozwiązaniach tego typu urządzenia kontrolujące zamki pracują w sposób autonomiczny. Oznacza to, iż cały proces uwierzytelniania dokonywany jest przez oprogramowanie mikroprocesora obsługującego zamek.

		% Na rynku dostępne są także rozwiązania sieciowe, bądź też takie, które umożliwiają konfigurację urządzeń w tryb zarówno autonomiczny, jak i sieciowy. Rozwiązania sieciowe charakteryzują się znacznym stopniem skomplikowania, zarówno pod względem architektonicznym (ilość i rodzaj potrzebnych komponentów sprzętowych)~\cite{racs5}, jak i konfiguracyjnym (trudność instalacji, konieczność modyfikacji istniejącej infrastruktury). Mogą oferować oddzielenie funkcjonalności czytnika dostępu od kontrolera, umożliwiając obsługę do kilkunastu czytników za pomocą jednego urządzenia kontrolującego~\cite{racs5}. Mimo możliwości dołączenia do kontrolera zamków bezprzewodowych, działanie całego systemu wciąż pozostaje w pewnym stopniu uzależnione od komunikacji przewodowej.

		% Ze względu na ilość komponentów sprzętowych, istniejące rozwiązania bywają drogie.