\section*{Wykaz ważniejszych oznaczeń i skrótów}

% Please add the following required packages to your document preamble:
% \usepackage{graphicx}
\begin{table}[htp]
    \centering
    \resizebox{\textwidth}{!}{%
    \begin{tabular}{lllll}
    \cline{1-2}
    \multicolumn{1}{|l|}{Pojęcie}       & \multicolumn{1}{l|}{Wyjaśnienie}                                                                                                                                       &  &  &  \\ \cline{1-2}
    \multicolumn{1}{|l|}{Punkt dostępu} & \multicolumn{1}{l|}{\begin{tabular}[c]{@{}l@{}}Fizyczne zabezpieczenie chroniące przed nieuprawnionym dostępem, \\ przykładowo: zamek, bramka\end{tabular}}            &  &  &  \\ \cline{1-2}
    \multicolumn{1}{|l|}{RFID}          & \multicolumn{1}{l|}{\begin{tabular}[c]{@{}l@{}}Technologia wykorzystująca fale radiowe w celu przesyłania danych\\ (ang. \textit{Radio-frequency identification})\end{tabular}} &  &  &  \\ \cline{1-2}
    \multicolumn{1}{|l|}{Karta}         & \multicolumn{1}{l|}{Karta zbliżeniowa RFID. Inne określenia: identyfikator, token}                                                                                     &  &  &  \\ \cline{1-2}
    \multicolumn{1}{|l|}{Kontroler}     & \multicolumn{1}{l|}{Komponent odpowiedzialny za zarządzanie układem zamka}                                                                                             &  &  &  \\ \cline{1-2}
                                        &                                                                                                                                                                        &  &  &
    \end{tabular}%
    }
    \end{table}