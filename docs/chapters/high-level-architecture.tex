\chapter{Projekt rozwiązania}
\label{chap:hl-arch}

Niniejszy rozdział opisuje projekt rozwiązania.

\section{Architektura systemu}
Ze względu na pełnione funkcje można wyodrębnić odrębne podsystemy. Są to:
\begin{enumerate}
        \item Podsystem sterowania zamkiem,
        \item Podsystem autoryzacji,
        \item Podsystem zarządzania.
\end{enumerate}

System składa się z dwóch głównych komponentów sprzętowych. Są to komponent sterujący zamkiem (inaczej: kontroler, układ zamka) oraz serwer główny. Komponenty te zostały krótko omówione w kolejnych punktach. Ogólna sprzętowa architektura systemu przedstawiona została na rysunku \ref{fig:hl-arch}.

\begin{figure}[]
        \includegraphics[width=\linewidth]{chapters/images/hl-arch.png}
        \caption{Architektura systemu}
        \label{fig:hl-arch}
\end{figure}

\subsection{Komponent sterujący zamkiem}
Komponent kontrolera składa się z mikrokontrolera ESP32 (\textbf{czy nawet plytki nie wiem jak to napisac}), czujnika ruchu oraz czytnika RFID. Mikrokontroler odpowiada za sterowanie peryferiami, zarządza ich zasilaniem, inicjuje i przeprowadza bezprzewodową komunikację z serwerem i steruje samym zamkiem na podstawie otrzymanych od serwera danych.

\textbf{tutaj obrazek z ukladem kontrolera}

\subsection{Serwer główny}
W ramach komponentu serwera działają dwa podsystemy funkcjonalne: podsystem autoryzacji, odpowiedzialny za podjęcie decyzji o przyznaniu lub odmowie dostępu na podstawie danych odebranych od podsystemu sterowania zamkiem, oraz podsystem zarządzania, odpowiedzialny za zbieranie oraz prezentację danych użytkownikowi.

Częścią serwera jest baza danych. Nie jest jednak konieczne, aby pozostawała ona fizycznie na tej samej maszynie.

\subsubsection{Podsystem autoryzacji}
Zadaniem podsystemu autoryzacji jest podjęcie decyzji o przyznaniu bądź odmowie dostępu na podstawie otrzymanych danych.

\subsubsection{Podsystem zarządzania}
Zadaniem podsystemu zarządzania jest umożliwienie użytkownikowi systemu wglądu do danych takich jak historia prób dostępu, zbiór zamków, kart oraz powiązań między nimi, oraz stan poszczególnych zamków.

\subsection{Komunikacja}

Komunikacja między kontrolerem a serwerem - wifi itd

\subsection{Architektura mikrokontrolera}

\section{Zasada działania}
Kontroler wbudowany w zamek pozostaje uśpiony do momentu wykrycia ruchu w pobliżu przez wbudowany czujnik ruchu. Po wybudzeniu nawiązuje bezpieczne połączenie z serwerem autoryzacji, jednocześnie zasilając czytnik RFID oraz oczekując na zbliżenie do niego karty. Gdy karta zostanie zbliżona, kontroler przesyła odczytany z niej numer identyfikacyjny do serwera, korzystając z nawiązanego wcześniej połączenia. Serwer podejmuje decyzję, którą jest przyznanie bądź odmowa dostępu, porównując odebrany numer identyfikacyjny z zawartością bazy danych, a następnie przesyła informację zwrotną do kontrolera. Jeżeli podjęto decyzję o przyznaniu dostępu, kontroler wysyła sygnał otwarcia zamka oraz sygnalizuje powodzenie za pomocą diody LED. Jeżeli podjęto decyzję o odmowie dostępu, kontroler sygnalizuje niepowodzenie za pomocą diody LED. Diagram sekwencji przedstawiony jest na rysunku \ref{fig:sequence1}.

\begin{figure}[]
        \includegraphics[width=\linewidth]{chapters/images/sequence1.png}
        \caption{Diagram sekwencji}
        \label{fig:sequence1}
\end{figure}

