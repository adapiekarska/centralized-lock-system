\chapter{Projekt rozwiązania}
\label{chap:hl-arch}

    Niniejszy rozdział przedstawia koncepcję systemu. Definiuje użytkowników oraz wymagania, opisuje i przedstawia jego budowę, a także zakładany sposób działania w kilku najbardziej prawdopodobnych sytuacjach.

    \section{Idea}
        Idea systemu przedstawiona została na rysnku \ref{fig:door}.

        \begin{figure}[h]
            \begin{center}
                \includegraphics[width=.9\linewidth]{chapters/images/door2.png}
                \caption{Idea systemu}
                \label{fig:door}
            \end{center}
        \end{figure}

    \section{Użytkownicy}
        Na potrzeby projektu rozwiązania zidentyfikowano dwóch użytkowników. Ich specyfikacja przedstawiona została w tabeli \ref{tbl:users}.

        \begin{table}
            \caption{Użytkownicy systemu}
            \centering
            \begin{subtable}[c]{\textwidth}
                \centering
                \begin{tabular}{|p{2cm}|p{10cm}|}
                    \hline USER\_1      & Użytkownik \\
                    \hline Opis         & Osoba posiadająca identyfikator, której celem jest uzyskanie dostępu do chronionego zamkiem pomieszczenia  \\
                    \hline Źródło       & Placeholder    \\
                    \hline
                \end{tabular}
                \label{tbl:usr1}
                \vspace{10mm}           
            \end{subtable}
        \quad%
            \begin{subtable}[c]{\textwidth}
                \centering
                \begin{tabular}{|p{2cm}|p{10cm}|}
                    \hline USER\_2      & Administrator \\
                    \hline Opis         & Osoba posiadająca uprawnienia administracyjne w systemie, mająca dostęp do oprogramowania zarządzającego \\
                    \hline Źródło       & Placeholder    \\
                    \hline
                \end{tabular}
                \label{tbl:usr2}
                \vspace{10mm}           
            \end{subtable}                
            \label{tbl:users}
        \end{table}

    \section{Wymagania funkcjonalne}

        W tabeli \ref{tbl:fnrq} przedstawiono wymagania funkcjonalne systemu.

        \begin{table}[h!]
            \caption{Wymagania funkcjonalne}
            \centering
            \begin{subtable}[c]{\textwidth}
                \centering
                \begin{tabular}{|p{2cm}|p{10cm}|}
                    \hline FNRQ\_1      & Kontrola dostępu do pomieszczeń  \\
                    \hline Opis         & Dopuszczanie do pomieszczeń użytkowników posiadających odpowiedni identyfikator i niedopuszczanie użytkowników nieposiadających odpowiedniego identyfikatora  \\
                    \hline Źródło       & Placeholder    \\
                    \hline
                \end{tabular}
                \label{tbl:fnrq1}
                \vspace{10mm}           
            \end{subtable}
        \quad%  
            \begin{subtable}[c]{\textwidth}
                \centering
                 \begin{tabular}{|p{2cm}|p{10cm}|}
                    \hline FNRQ\_2      & Przeglądanie historii prób dostępu do pomieszczeń  \\
                    \hline Opis         & Dostęp do listy dokonanych w przeszłości prób dostępu zakończonych zarówno sukcesem, jak i porażką \\
                    \hline Źródło       & Placeholder    \\
                    \hline
                \end{tabular}
                \label{tbl:fnrq2}
                \vspace{10mm}           
            \end{subtable}
        \quad%   
            \begin{subtable}[c]{\textwidth}
                \centering
                 \begin{tabular}{|p{2cm}|p{10cm}|}
                    \hline FNRQ\_3      & Dodawanie identyfikatorów  \\
                    \hline Opis         & Dodawanie identyfikatorów wraz z przyznaniem dostępu do wybranej grupy zamków \\
                    \hline Źródło       & Placeholder    \\
                    \hline
                \end{tabular}
                \label{tbl:fnrq3}
                \vspace{10mm}           
            \end{subtable}
        \quad%  
            \begin{subtable}[c]{\textwidth}
                \centering
                 \begin{tabular}{|p{2cm}|p{10cm}|}
                    \hline FNRQ\_4      & Przeglądanie identyfikatorów powiązanych z danym zamkiem  \\
                    \hline Opis         & Dostęp do listy powiązań między identyfikatorami a zamkami \\
                    \hline Źródło       & Placeholder    \\
                    \hline
                \end{tabular}
                \label{tbl:fnrq4}
                \vspace{10mm}           
            \end{subtable}
        \quad%
            \begin{subtable}[c]{\textwidth}
                \centering
                 \begin{tabular}{|p{2cm}|p{10cm}|}
                    \hline FNRQ\_5      & Blokowanie dostępu do pomieszczeń dla wybranego identyfikatora  \\
                    \hline Opis         & Manualny wybór opcji czasowego usunięcia powiązania wybranego identyfikatora z wybraną grupą zamków \\
                    \hline Źródło       & Placeholder    \\
                    \hline
                \end{tabular}
                \label{tbl:fnrq5}
                \vspace{10mm}           
            \end{subtable}
        \label{tbl:fnrq}
        \end{table}

        \pagebreak

        \begin{table}
        \ContinuedFloat
        \caption{Wymagania funkcjonalne, c.d.}
            \begin{subtable}[c]{\textwidth}
                \centering
                 \begin{tabular}{|p{2cm}|p{10cm}|}
                    \hline FNRQ\_6      & Dostęp do grupy pomieszczeń za pomocą jednego identyfikatora  \\
                    \hline Opis         & Ustawienie powiązań identyfikatorów i zamków w taki sposób, aby możliwy był dostęp do grupy zamków za pomocą jednego identyfikatora \\
                    \hline Źródło       & Placeholder    \\
                    \hline
                \end{tabular}
                \label{tbl:fnrq6}
                \vspace{10mm}           
            \end{subtable}
        \quad%
            \begin{subtable}[c]{\textwidth}
                \centering
                 \begin{tabular}{|p{2cm}|p{10cm}|}
                    \hline FNRQ\_7      & Dostęp do pomieszczenia za pomocą grupy identyfikatorów  \\
                    \hline Opis         & Ustawienie powiązań identyfikatorów i zamków w taki sposób, aby możliwy był dostęp do jednego zamka za pomocą grupy identyfikatorów \\
                    \hline Źródło       & Placeholder    \\
                    \hline
                \end{tabular}
                \label{tbl:fnrq7}
                \vspace{10mm}           
            \end{subtable}
        \quad%
            \begin{subtable}[c]{\textwidth}
                \centering
                 \begin{tabular}{|p{2cm}|p{10cm}|}
                    \hline FNRQ\_8      & Sygnalizacja przyznania lub odmowy dostępu  \\
                    \hline Opis         & Powiadamianie użytkownika o podjętej przez system decyzji \\
                    \hline Źródło       & Placeholder    \\
                    \hline
                \end{tabular}
                \label{tbl:fnrq8}
                \vspace{10mm}           
            \end{subtable}
        \quad%
            \begin{subtable}[c]{\textwidth}
                \centering
                 \begin{tabular}{|p{2cm}|p{10cm}|}
                    \hline FNRQ\_9      & Sygnalizacja stanu naładowania baterii w układzie zamka  \\
                    \hline Opis         & Okresowe powiadamianie administratora systemu o bieżącym stanie naładowania baterii \\
                    \hline Źródło       & Placeholder    \\
                    \hline
                \end{tabular}
                \label{tbl:fnrq9}
                \vspace{10mm}           
            \end{subtable}
            \label{tbl:fnrq_2}
        \end{table}

        \section{Wymagania pozafunkcjonalne}

            W tabeli \ref{tbl:xxrq} przedstawiono wymagania pozafunkcjonalne systemu.

            \begin{table}
            \caption{Wymagania pozafunkcjonalne}
            \centering
            \begin{subtable}[c]{\textwidth}
                \centering
                \begin{tabular}{|p{2cm}|p{10cm}|}
                    \hline XXRQ\_1      & Długość czasu pracy na baterii równa minimum 1 rok  \\
                    \hline Opis         & Placeholder  \\
                    \hline Źródło       & Placeholder    \\
                    \hline
                \end{tabular}
                \label{tbl:xxrq1}
                \vspace{10mm}           
            \end{subtable}
            \label{tbl:xxrq}
        \end{table}


        \section{Sposób działania}
            Działanie systemu opiera się na współpracy układu zamka z serwerem w celu zapewnienia poprawnej autoryzacji identyfikatorów w zamkach. Funkcjonalność autoryzacji realizowana jest wyłącznie po stronie serwera. Układ zamka, będący urządzeniem klienckim, jest jedynie pośrednikiem przekazującym dane od użytkownika do serwera.

            W celu osiągnięcia największej możliwej wydajności energetycznej czytnik RFID oraz mikrokontroler przez większą część czasu pozostają w stanie uśpienia. Zadaniem czujnika ruchu, zasilanego przez cały czas, jest odpowiednio wczesne wykrycie zbliżającego się użytkownika i wybudzenie mikrokontrolera, który z kolei jest odpowiedzialny za zasilenie czytnika RFID oraz nawiązanie połączenia z serwerem. Jeśli operacja nawiązania połączenia przebiegnie pomyślnie, a do czytnika przyłożony zostanie identyfikator, rozpoczyna się proces przekazywania danych odczytanych z identyfikatora do serwera w celu autoryzacji identyfikatora. Należy zwrócić uwagę na niekorzystne z punktu widzenia systemu warunki, które mogą zajść w trakcie procesu:

            \begin{enumerate}
                \item
                    Jeżeli identyfikator nie zostanie przyłożony w przeciągu 10~sek. od momentu wybudzenia czytnika, mikrokontroler ponownie wprowadza czytnik oraz samego siebie w stan uśpienia.
                \item
                    Jeżeli połączenie z serwerem nie może zostać nawiązane w czasie \(t + 5~sek.\), gdzie \(t\) jest zmiennym czasem upływającym od momentu podjęcia próby nawiązania połączenia z serwerem do momentu zakończenia odczytu danych z identyfikatora, mikrokontroler sygnalizuje użytkownikowi błąd połączenia, po czym wprowadza się w stan uśpienia.
            \end{enumerate}

            Jeżeli żadna z wymienionych wyżej niekorzystnych sytuacji nie wystąpi i dane zostaną pomyślnie przesłane do serwera, serwer podejmuje decyzję o przyznaniu bądź odmowie dostępu. Dokonuje tego po wysłaniu stosownego zapytania do bazy danych, a następnie wysyła potwierdzenie lub odmowę do mikrokontrolera. Mikrokontroler w sposób wizualny sygnalizuje decyzję użytkownikowi, a jeżeli była ona pomyślna, dodatkowo wysyła sygnał otwierający zamek. Niezależnie od decyzji serwera, wpis o próbie dostępu zostaje zapisany w bazie danych, skąd może być pobrany przez podsystem zarządzania w celu prezentacji danych administratorowi systemu.

            Opisane wyżej mechanizmy zostały szerzej ukazane na diagramach sekwencji. Diagramy \ref{fig:sequence1}-\ref{fig:sequence3} dotyczą podsystemu sterowania zamkiem oraz podsystemu autoryzacji, natomiast diagram \ref{fig:sequence4} dotyczy podsystemu zarządzania. Diagram \ref{fig:sequence1} przedstawia proces autoryzacji identyfikatora użytkownika w przypadku najbardziej pomyślnego scenariusza. Diagram \ref{fig:sequence2} ukazuje przepływ sterowania pomiędzy elementami systemu w sytuacji, gdy użytkownik zostanie wykryty, ale identyfikator nie zostanie przyłożony do czytnika w zadanym przedziale czasu. Diagram \ref{fig:sequence3} przedstawia przepływ sterowania pomiędzy elementami systemu w sytuacji, gdy niemożliwe jest nawiązanie połączenia z serwerem. Diagram \ref{fig:sequence4} ukazuje przepływ sterowania pomiędzy warstwami podsystemu zarządzania w sytuacji żądania dostępu do historii prób dostępu przez administratora systemu.

            Należy zwrócić uwagę na możliwość wystąpienia również innych sytuacji niekorzystnych, takich jak błąd połączenia z bazą danych lub \textbf{co jeszcze?}. Ich obsługa jest pomijalna z punktu widzenia współpracy komponentów systemu, dlatego nie została uwzględniona na diagramach.

            \begin{figure}[]
                \includegraphics[width=\linewidth]{chapters/images/sequence1.png}
                \caption{Przepływ sterowania w procesie autoryzacji identyfikatora, pomyślny scenariusz}
                \label{fig:sequence1}
            \end{figure}

            \begin{figure}[]
                \includegraphics[width=\linewidth]{chapters/images/sequence2.png}
                \caption{Przepływ sterowania w sytuacji wykrycia użytkownika nie prezentującego identyfikatora}
                \label{fig:sequence2}
            \end{figure}

            \begin{figure}[]
                \includegraphics[width=\linewidth]{chapters/images/sequence3.png}
                \caption{Przepływ sterowania w sytuacji braku możliwości nawiązania połączenia z serwerem}
                \label{fig:sequence3}
            \end{figure}

            \begin{figure}[]
                \centering
                \includegraphics[width=.7\linewidth]{chapters/images/sequence4.png}
                \caption{Przepływ sterowania w procesie żądania dostępu do historii prób dostępu}
                \label{fig:sequence4}
            \end{figure}

            \textbf{dodać scenariusz dodawania nowego identyfikatora}

        \section{Architektura systemu}

            \subsection{Podsystemy}

                System podzielony został na podsystemy realizujące określone funkcjonalności i istniejące w ramach komponentów sprzętowych. Poniżej wylistowano wyodrębnione podsystemy wraz z ich najważniejszymi funkcjami. 

                \begin{enumerate}

                    \item Podsystem sterowania zamkiem

                        Jest odpowiedzialny za odczyt danych identyfikatora użytkownika oraz przekazanie ich do podsystemu autoryzacji, zarządzanie zasilaniem elementów układu zamka oraz zarządzanie samym zamkiem.

                    \item Podsystem autoryzacji

                        Jego zadaniem jest podjęcie decyzji o przyznaniu bądź odmowie dostępu na podstawie otrzymanych od podsystemu sterowania zamkiem danych. Komunikuje się z podsystemem sterowania zamkiem oraz bazą danych.

                    \item Podsystem zarządzania

                        Jego zadaniem jest umożliwienie administratorowi systemu wglądu do danych takich jak historia prób dostępu, zbiór identyfikatorów, zamków, oraz powiązań między nimi, a także stan poszczególnych zamków. Dzięki niemu możliwa jest konfiguracja rozpoznawanych przez system identyfikatorów i zamków oraz manualne przyznawanie dostępu poszczególnym identyfikatorom.

                    \item Baza danych

                        Ze względu na to, że stanowi część zarówno podsystemu autoryzacji, jak i podsystemu zarządzania, traktowana jest jako osobny element systemu niepodlegający żadnemu innemu podsystemowi. Baza danych przechowuje dane dotyczące poszczególnych identyfikatorów i zamków zarejestrowanych w systemie, powiązań pomiędzy nimi oraz dokonanych w przeszłości prób dostępu zakończonych zarówno sukcesem jak i porażką. Schemat bazy danych przedstawiony został na rysunku \ref{fig:schema}.

                        \begin{figure}
                            \centering
                            \includegraphics[width=.8\linewidth]{chapters/images/schema.png}
                            \caption{Schemat bazy danych}
                            \label{fig:schema}
                        \end{figure}

                \end{enumerate}

            \subsection{Komponenty sprzętowe}

                Podsystemy sterowania zamkiem, autoryzacji i zarządzania oraz baza danych istnieją w ramach dwóch komponentów sprzętowych. Komponenty te to:
                \begin{enumerate}
                    \item Komponent sterujący zamkiem, inaczej nazywany układem zamka - w jego obrębie zlokalizowany jest podsystem sterowania zamkiem,
                    \item Komponent serwera - w jego obrębie zlokalizowane są podsystemy autoryzacji, zarządzania oraz baza danych.
                \end{enumerate}

                Komponenty sprzętowe zostały krótko omówione w kolejnych punktach. Sprzętowa architektura systemu z podziałem na komponenty oraz przynależne im podsystemy przedstawiona została na rysunku \ref{fig:hl-arch}.

                \begin{figure}[]
                    \includegraphics[width=\linewidth]{chapters/images/hl-arch3.png}
                    \caption{Architektura systemu}
                    \label{fig:hl-arch}
                \end{figure}

                \begin{enumerate}

                    \item Komponent sterujący zamkiem
                    
                        Jedynym podsystemem zlokalizowanym w obrębie tego komponentu jest podsystem sterowania zamkiem (patrz rysunek \ref{fig:hl-arch}).
                        Komponent sterujący zamkiem tworzony jest przez następujące bloki funkcjonalne:

                        \begin{itemize}
                            \item Mikrokontroler

                                Odpowiada za sterowanie peryferiami, zarządzaniem ich zasilaniem, inicjację i przeprowadzenie bezprzewodowej komunikacji z serwerem i sterowanie samym zamkiem.

                            \item Czujnik ruchu

                                Jego jedynym zadaniem jest wykrycie zbliżającego się użytkownika.

                            \item Czytnik RFID

                                Stanowi interfejs pomiędzy użytkownikiem a systemem.

                        \end{itemize}

                        Diagram \textbf{jak się nazywa taki diagram?} komponentu sterującego zamkiem został przedstawiony na rysunku \ref{fig:lock-arch}.

                        \begin{figure}
                            \centering
                            \includegraphics[width=0.7\textwidth]{chapters/images/lock.png}
                            \caption{Diagram komponentu sterującego zamkiem}
                            \label{fig:lock-arch}
                        \end{figure}

                    \item Komponent serwera

                \end{enumerate}

                    

