\chapter{Projekt rozwiązania}
\label{chap:hl-arch}

        Niniejszy rozdział opisuje koncept rozwiązania, który powstał w ramach pracy. Przedstawia schematyczną budowę systemu oraz nakreśla zakładany sposób działania.

        Ogólny zarys konceptu działania systemu przedstawiony jest na rysnku \ref{fig:door}.

        \begin{figure}[]
                \includegraphics[width=\linewidth]{chapters/images/door2.png}
                \caption{Koncept systemu kontroli dostępu}
                \label{fig:door}
        \end{figure}

        \section{Architektura systemu}
                W ramach systemu można wyodrębnić następujące podsystemy:
                \begin{enumerate}
                        \item Podsystem sterowania zamkiem,
                        \item Podsystem autoryzacji,
                        \item Podsystem zarządzania.
                \end{enumerate}

                Podsystemy te istnieją w ramach dwóch komponentów sprzętowych. Są to:
                \begin{enumerate}
                        \item Komponent sterujący zamkiem (inaczej: kontroler, układ zamka),
                        \item Komponent serwera.
                \end{enumerate}

                Komponenty te zostały krótko omówione w kolejnych punktach. Ogólna sprzętowa architektura systemu z podziałem na komponenty sprzętowe oraz przynależne im podsystemy przedstawiona została na rysunku \ref{fig:hl-arch}.

                \begin{figure}[]
                        \includegraphics[width=\linewidth]{chapters/images/hl-arch3.png}
                        \caption{Architektura systemu}
                        \label{fig:hl-arch}
                \end{figure}

                \subsection{Komponent sterujący zamkiem}
                        Komponent sterujący zamkiem składa się z mikrokontrolera, czujnika ruchu oraz czytnika RFID. Mikrokontroler odpowiada za sterowanie peryferiami, zarządza ich zasilaniem, inicjuje i przeprowadza bezprzewodową komunikację z serwerem i steruje samym zamkiem na podstawie otrzymanych od serwera danych. Podsystem sterowania zamkiem zlokalizowany jest w całości w tym komponencie (patrz rysunek \ref{fig:hl-arch}). Architektura układu zamka została przedstawiona na rysunku \ref{fig:lock-arch}. 

                        \begin{figure}
                                \centering
                                \includegraphics[width=0.7\textwidth]{chapters/images/lock.png}
                                \caption{Architektura układu zamka}
                                \label{fig:lock-arch}
                        \end{figure}

                \subsection{Serwer główny}
                        W ramach komponentu serwera działają dwa podsystemy funkcjonalne: podsystem autoryzacji, odpowiedzialny za podjęcie decyzji o przyznaniu lub odmowie dostępu na podstawie danych odebranych od podsystemu sterowania zamkiem, oraz podsystem zarządzania, odpowiedzialny za zbieranie oraz prezentację danych użytkownikowi.

                \subsubsection{Baza danych}
                        Częścią komponentu serwera jest baza danych. Nie jest jednak konieczne, aby pozostawała ona fizycznie na tej samej maszynie. W przypadku całkowitego rozdzielenia serwera danych od serwera autoryzacji skalowalność systemu znacząco wzrośnie.

                \subsubsection{Podsystem autoryzacji}
                        Zadaniem podsystemu autoryzacji jest podjęcie decyzji o przyznaniu bądź odmowie dostępu na podstawie otrzymanych danych. Podsystem komunikuje się z bazą danych w celu uzyskania informacji na temat autoryzowanych kart.

                \subsubsection{Podsystem zarządzania}
                        Zadaniem podsystemu zarządzania jest umożliwienie użytkownikowi systemu wglądu do danych takich jak historia prób dostępu, zbiór zamków, kart oraz powiązań między nimi, oraz stan poszczególnych zamków. Dzięki niemu możliwa jest konfiguracja rozpoznawanych przez system kart, zamków oraz manualne przyznawanie dostępu poszczególnym identyfikatorom i punktom dostępu.

        \section{Zasada działania}
                Kontroler wbudowany w zamek pozostaje uśpiony do momentu wykrycia ruchu w pobliżu przez wbudowany czujnik ruchu. Po wybudzeniu nawiązuje bezpieczne połączenie z serwerem autoryzacji, jednocześnie zasilając czytnik RFID oraz oczekując na zbliżenie do niego karty. Gdy karta zostanie zbliżona, kontroler przesyła odczytany z niej numer identyfikacyjny do serwera, korzystając z nawiązanego wcześniej połączenia. Serwer podejmuje decyzję, którą jest przyznanie bądź odmowa dostępu, porównując odebrany numer identyfikacyjny z zawartością bazy danych, a następnie przesyła informację zwrotną do kontrolera. Jeżeli podjęto decyzję o przyznaniu dostępu, kontroler wysyła sygnał otwarcia zamka oraz sygnalizuje powodzenie. Jeżeli podjęto decyzję o odmowie dostępu, kontroler sygnalizuje niepowodzenie. Diagram sekwencji przedstawiony jest na rysunku \ref{fig:sequence1}.

                \begin{figure}[]
                        \includegraphics[width=\linewidth]{chapters/images/sequence1.png}
                        \caption{Diagram sekwencji ukazujący zasadę działania systemu}
                        \label{fig:sequence1}
                \end{figure}

