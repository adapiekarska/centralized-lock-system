\section*{Streszczenie}
	Postępująca miniaturyzacja systemów wbudowanych wynikająca z wykładniczego rozwoju technologii półprzewodnikowych oraz rosnąca ich dostępność sprawiają, że systemy informatyczne znajdują coraz szersze zastosowanie w wielu dziedzinach. Nowoczesne i wydajne systemy stopniowo zastępują tradycyjne rozwiązania sprzed ery informatyzacji. Nowe możliwości niosą ze sobą jednak równie wiele wyzwań, szczególnie z zakresu bezpieczeństwa. Jedną z dziedzin o szerokich perspektywach rozwoju jest bezpieczeństwo przestrzeni wraz z ich uzytkownikami. Niniejsza praca opisuje projekt i implementację bezprzewodowego systemu dostępu do pomieszczeń. Omawia architekturę rozwiązania z uwzględnieniem poszczególnych podsystemów, przedstawia ciekawe aspekty realizacji projektu oraz jego rezultaty. Ponadto, prezentuje zagadnienia związane z bezpieczeństwem oraz wydajnością energetyczną bezprzewodowych systemów opartych na mikrokontrolerach.

	\textbf{TBD - rozszerzyć}

	\textbf{Słowa kluczowe:} zamek elektroniczny, mikrokontroler, kontrola dostępu, WiFi, RFID, sieć bezprzewodowa, autoryzacja

	\textbf{Dziedzina nauki i techniki, zgodnie z wymogami OECD:}

\section*{Abstract}
	Postępująca miniaturyzacja systemów wbudowanych wynikająca z wykładniczego rozwoju technologii półprzewodnikowych oraz rosnąca ich dostępność sprawiają, że systemy informatyczne znajdują coraz szersze zastosowanie w wielu dziedzinach. Nowoczesne i wydajne systemy stopniowo zastępują tradycyjne rozwiązania sprzed ery informatyzacji. Nowe możliwości niosą ze sobą jednak równie wiele wyzwań, szczególnie z zakresu bezpieczeństwa. Jedną z dziedzin o szerokich perspektywach rozwoju jest bezpieczeństwo przestrzeni wraz z ich uzytkownikami. Niniejsza praca opisuje projekt i implementację bezprzewodowego systemu dostępu do pomieszczeń. Omawia architekturę rozwiązania z uwzględnieniem poszczególnych podsystemów, przedstawia ciekawe aspekty realizacji projektu oraz jego rezultaty. Ponadto, prezentuje zagadnienia związane z bezpieczeństwem oraz wydajnością energetyczną bezprzewodowych systemów opartych na mikrokontrolerach.

	\textbf{Keywords:}