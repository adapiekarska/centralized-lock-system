\section*{Streszczenie}
	Systemy wykorzystujące urządzenia elektroniczne od wielu lat stosowane są we wszystkich dziedzinach ludzkiego życia. Rozwój technologii bezprzewodowych oraz postępująca miniaturyzacja urządzeń elektronicznych sprawiają, że systemy te stają się nowocześniejsze, bezpieczniejsze i wydajniejsze. Jednym z celów, dla których stosowane są tego typu systemy jest zapewnienie bezpieczeństwa ludziom oraz mieniu. Niniejsza praca opisuje projekt i implementację bezprzewodowego systemu dostępu do pomieszczeń. Omawia architekturę rozwiązania z uwzględnieniem poszczególnych podsystemów, przedstawia ciekawe aspekty realizacji projektu oraz jego rezultaty. Ponadto, prezentuje zagadnienia związane z bezpieczeństwem oraz wydajnością energetyczną bezprzewodowych systemów opartych na mikrokontrolerach.

	\textbf{Słowa kluczowe:} zamek elektroniczny, mikrokontroler, kontrola dostępu, WiFi, RFID, sieć bezprzewodowa, autoryzacja

	\textbf{Dziedzina nauki i techniki, zgodnie z wymogami OECD:}

\section*{Abstract}
	Systemy wykorzystujące urządzenia elektroniczne od wielu lat stosowane są we wszystkich dziedzinach ludzkiego życia. Rozwój technologii bezprzewodowych oraz postępująca miniaturyzacja urządzeń elektronicznych sprawiają, że systemy te stają się nowocześniejsze, bezpieczniejsze i wydajniejsze. Jednym z celów, dla których stosowane są tego typu systemy jest zapewnienie bezpieczeństwa ludziom oraz mieniu. Niniejsza praca opisuje projekt i implementację bezprzewodowego systemu dostępu do pomieszczeń. Omawia architekturę rozwiązania z uwzględnieniem poszczególnych podsystemów, przedstawia ciekawe aspekty realizacji projektu oraz jego rezultaty. Ponadto, prezentuje zagadnienia związane z bezpieczeństwem oraz wydajnością energetyczną bezprzewodowych systemów opartych na mikrokontrolerach.

	\textbf{Keywords:}