\section*{Streszczenie}

	Rosnąca dostępność i postępująca miniaturyzacja układów elektronicznych spowodowana wykładniczym rozwojem technologii półprzewodnikowych sprawiają, że systemy o nie oparte zyskują coraz większą popularność. Rozwiązania sprzed ery informatyzacji są stopniowo wypierane przez ich nowoczesne odpowiedniki, jednak wykorzystanie nowych możliwości wiąże się także z koniecznością zmierzenia się z wieloma wyzwaniami, zwłaszcza w kwestii bezpieczeństwa. Jedną z dziedzin, w której problemy te są szczególnie istotne, jest kontrola dostępu w obiektach budowlanych. Systemy kontroli dostępu muszą implementować stosowne mechanizmy bezpieczeństwa, ponieważ udany atak na tego typu system może doprowadzić do ekspozycji wrażliwych zasobów lub naruszenia prywatności osób.

	Niniejsza praca opisuje projekt i implementację systemu kontroli dostępu do pomieszczeń, w którym za przeprowadzenie procesu autoryzacji odpowiedzialny jest centralny serwer, a komunikacja pomiędzy zamkami a serwerem odbywa się bezprzewodowo. Praca przedstawia dziedzinę problemu, dokonuje przeglądu istniejących rozwiązań, a także omawia projekt rozwiązania oraz ciekawsze zagadnienia oraz rezultaty procesu implementacji zarówno części serwerowej, jak i klienckiej. Ponadto prezentuje zagadnienia związane z bezpieczeństwem oraz wydajnością energetyczną bezprzewodowych systemów opartych na mikrokontrolerach.

	\textbf{Słowa kluczowe:} zamek elektroniczny, mikrokontroler, kontrola dostępu, WiFi, RFID, sieci bezprzewodowe, autoryzacja

	\textbf{Dziedzina nauki i techniki, zgodnie z wymogami OECD:} Nauki inżynieryjne i techniczne, Systemy automatyzacji i kontroli

\section*{Abstract}

	Increasing availability and progressing miniaturization of electronic circuits caused by exponential development of semiconductor technologies give rise to electronic systems' popularity. Pre-computerization era solutions are being replaced by their present-day counterparts. However, taking advantage of new possibilities makes it necessary to face numerous challenges, especially in the field of security. Such problems are particularly relevant in the area of access control within buildings. Access control systems must implement appropriate security mechanisms, as successful attack on these kind of systems may lead to exposing of vulnerable resources or privacy violation.

	This thesis aims to showcase the design and implementation of the access control system, in which the authorization process is performed solely by the server and the communication between locks and the server is done wirelessly. The thesis presents the problem domain, overviews some of the existing solutions, discusses the proposed design and selected issues as well as obtained results of the implementation process for both server and client parts. Besides, it addresses issues related to security and energy efficiency of the microcontroller-based wireless systems.

	\textbf{Keywords:} electronic lock, microcontroller, access control, WiFi, RFID, wireless networks, authorization